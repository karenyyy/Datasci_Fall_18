\section{Background}

In the 21st century, digitalization and automation of industry, education, business and entertainment cultivate our dependence of various softwares that facilitate our lives. The massive adoption of technology in basically every aspect of society nowadays calls for the professional knowledge of software engineers to contribute to the benefit of the majority as well as the commitment  to ethical principles from these professionals to prevent them from applying expert skills to commit practices that are detrimental to the public. In accordance with their commitment to the health, safety and welfare of the public, software engineers shall adhere to the following Eight Principles\cite{Gotterbarn:1999:PST:308769.308770}: 

\begin{itemize}
	\item Public
	\item Client and Employer
	\item Product
	\item Judgment
	\item Management
	\item Profession
	\item Colleagues
	\item Self
\end{itemize}



\section{Ethic Principles}

\subsection{Public}

During the design, development, testing and maintenance of softwares for enterprises as well as public institutions and governmental organizations, public interest should always be the top concern for software engineers. Because technology is a double-edged sword in the era when basically every individual is associated with it and entrust their private information onto platforms supported by it, which makes the role as software engineers more essential and honorable than ever in history. However, a software professional lack of moral righteousness could be harmful to the public benefit, for instance, engineers who intentionally develop malwares to attack systems of companies and institutions, deliberately conceal software malfunctioning and potential bugs that could lead to unimaginable economical cost, etc. All softwares should be developed for achieving the greater good. Thus, engineers should always approve software only if they have a well-founded belief that it is safe, meets specifications, passes appropriate tests, and does not diminish quality of life, diminish privacy or harm the environment\cite{Gotterbarn:1999:PST:308769.308770}. 


\subsection{Client and Employer}

Besides public interest, software engineers are employees of certain organizations after all. Hence consistent with the benefit of the whole society, the interest of their clients (users) and employers should be their major concern as well. Unlike the marketing personnel in a company who are just responsible for selling the products to clients, engineers are a group of professionals who are actually  aware of every technical detail of the products, including the advantages and limitations, of which they ought to give their clients a full disclosure. To their employers, engineers need to pay serious attention to the protection of intellectual property rights. Using any pieces of illegally downloaded softwares is considered a violation of copy rights. And it's exactly this kind of indiscretion that could likely get themselves as well as their employers involved in lawsuits. Also, to act as responsible employees, any confidential work done while working for that institution should always be kept classified, unless a higher ethical concern is being compromised\cite{Gotterbarn:1999:PST:308769.308770}. In that case, first inform the employers of the ethical concern to protect the interest of employers as much as possible, if fails, then go to relevant authorities as whistle-blowers.



\subsection{Product}

Competent Engineers are all perfectionists.  Always striving to develop user-friendly softwares with high quality, multiple platform compatibility, high processing speed and reasonable cost is not only the professional standard that is expected from engineers, but also their motto as technological experts. To be specific, before committing to a project, since engineers know the technical details inside and out, they ought to provide an uncertainty assessment of estimates of realistic quantitative estimates of cost, scheduling, personnel, quality and outcomes on any project on which they work or propose to work\cite{Gotterbarn:1999:PST:308769.308770}.
 During the development process, engineers who are in charge of the project need to mark down each milestone in each phrase of implementation to ensure the final product delivery in timely manner. When it's time to release and deploy, other than solely focusing on the technical details, responsible engineers should always double confirm the completion of a well-written document or instruction, possible legal issues, terms of service, approvals from relevant authority departments, etc. In the after-sales services, treat all forms of software maintenance with the same professionalism as new development\cite{Gotterbarn:1999:PST:308769.308770}.
 
 
 \subsection{Judgment}
 
 As employees of certain organizations, engineers need to be loyal to their employers. Exercising their righteous judgment to avoid any possible conflicts of interest and not engage in deceptive financial practices such as bribery, double billing, or other improper financial practices\cite{Gotterbarn:1999:PST:308769.308770} is  the exact illustration of such loyalty as well as integrity as an independent professional.
 
 
 \subsection{Management}
 
 As senior engineers who are promoted to managerial positions, besides the responsibility to employers, there are also several basic rules that they are expected to commit in order to establish well-designed management strategies as good leaders to their teams. During recruiting, any kind of favoritism such as nepotism or under-the-table bribery  are by no means acceptable. All recruitment decisions must be made fairly and justly based solely on the candidates' qualifications. And when delegating, to ensure maximum efficiency of successful delivery, team leaders should assign work only after taking into account appropriate contributions of education and experience tempered with a desire to further that education and experience and offer fair and just remuneration afterwards\cite{Gotterbarn:1999:PST:308769.308770}.
 
 
 \subsection{Profession}
 
 Software engineers are not only the practitioner of their fields, but also the evangelist who 
 propagate software engineering knowledge by appropriate participation in professional organizations, meetings and publications, without promoting their own interest  at the expense of the profession during which process\cite{Gotterbarn:1999:PST:308769.308770}. Hence any claims that come from them should be able to stand up for scrutiny from their peers as well as the general public. It is considered an unethical action of professionals to deliberately release vague, misleading or even worse, false statements about any product they are in charge of.
 
 
 
 \subsection{Colleagues}
 
 Almost in all professional fields including software engineering, peer-reviewing is encouraged and considered an essential process of getting validation and credibility of one's work. A competent engineer should be a team player who's ready to work with equally-qualified colleagues any time when requested, which includes assisting their peers with any technical problems within their capability when necessary, taking constructive opinions and advices from peers, accepting complaints from peers and reflecting on any possible mistakes made, etc. And in order to achieve an actual win-win situation when teaming up with peers, what a responsible engineer should always keep in mind is to credit fully the work of others and refrain from taking undue credit\cite{Gotterbarn:1999:PST:308769.308770} to avoid any intellectual property rights conflicts in the foreseeable future. Intentionally sabotage the work of colleagues is unethical as well for that the interest of public, employers and clients should always come first during professional practices. Such action is clearly an infringement of other individual's lawful rights, and detrimental to the potential benefits of employers in the long run.
 
 
 \subsection{Self}
 
 Being a software engineer, the passion to pursue technical advancement to make the world a better place is a life-long mission. But in some cases, engineers could get too carried away with regard to the technical details and overlook other aspects that are essential in their career as software engineers. For instance the financial cost to build their project, legal issues associated with the release and deployment of their product, and most important of all, moral rules that they should hold on to throughout their entire career for the greater good of society, which is one of the major criteria that determines the integrity of truly qualified software engineer.
 

 
 
 
 \section{Conclusion}
 
Among various lines of work, software engineering gains astonishing popularity in the recent decades and becomes an essential and lucrative career. But being a software engineer means far more than the profits earned and reputation gained, it also comes with loyalty to employers, responsibility to the general public, clients, colleagues and themselves. Software engineers shall commit themselves to making the analysis, specification, design, development, testing and maintenance of software a beneficial and respected profession\cite{Gotterbarn:1999:PST:308769.308770}. Hence the eight ethical principles discussed in this paper should not only act as the standard guidance in the professional practice of all software engineers, but also their life-long motto to which they obliged to adhere in order to make a successful career in software engineering.
  