
% Default to the notebook output style

    


% Inherit from the specified cell style.




    
\documentclass[11pt]{article}

    
    
    \usepackage[T1]{fontenc}
    % Nicer default font (+ math font) than Computer Modern for most use cases
    \usepackage{mathpazo}

    % Basic figure setup, for now with no caption control since it's done
    % automatically by Pandoc (which extracts ![](path) syntax from Markdown).
    \usepackage{graphicx}
    % We will generate all images so they have a width \maxwidth. This means
    % that they will get their normal width if they fit onto the page, but
    % are scaled down if they would overflow the margins.
    \makeatletter
    \def\maxwidth{\ifdim\Gin@nat@width>\linewidth\linewidth
    \else\Gin@nat@width\fi}
    \makeatother
    \let\Oldincludegraphics\includegraphics
    % Set max figure width to be 80% of text width, for now hardcoded.
    \renewcommand{\includegraphics}[1]{\Oldincludegraphics[width=.8\maxwidth]{#1}}
    % Ensure that by default, figures have no caption (until we provide a
    % proper Figure object with a Caption API and a way to capture that
    % in the conversion process - todo).
    \usepackage{caption}
    \DeclareCaptionLabelFormat{nolabel}{}
    \captionsetup{labelformat=nolabel}

    \usepackage{adjustbox} % Used to constrain images to a maximum size 
    \usepackage{xcolor} % Allow colors to be defined
    \usepackage{enumerate} % Needed for markdown enumerations to work
    \usepackage{geometry} % Used to adjust the document margins
    \usepackage{amsmath} % Equations
    \usepackage{amssymb} % Equations
    \usepackage{textcomp} % defines textquotesingle
    % Hack from http://tex.stackexchange.com/a/47451/13684:
    \AtBeginDocument{%
        \def\PYZsq{\textquotesingle}% Upright quotes in Pygmentized code
    }
    \usepackage{upquote} % Upright quotes for verbatim code
    \usepackage{eurosym} % defines \euro
    \usepackage[mathletters]{ucs} % Extended unicode (utf-8) support
    \usepackage[utf8x]{inputenc} % Allow utf-8 characters in the tex document
    \usepackage{fancyvrb} % verbatim replacement that allows latex
    \usepackage{grffile} % extends the file name processing of package graphics 
                         % to support a larger range 
    % The hyperref package gives us a pdf with properly built
    % internal navigation ('pdf bookmarks' for the table of contents,
    % internal cross-reference links, web links for URLs, etc.)
    \usepackage{hyperref}
    \usepackage{longtable} % longtable support required by pandoc >1.10
    \usepackage{booktabs}  % table support for pandoc > 1.12.2
    \usepackage[inline]{enumitem} % IRkernel/repr support (it uses the enumerate* environment)
    \usepackage[normalem]{ulem} % ulem is needed to support strikethroughs (\sout)
                                % normalem makes italics be italics, not underlines
    \usepackage{mathrsfs}
    

    
    
    % Colors for the hyperref package
    \definecolor{urlcolor}{rgb}{0,.145,.698}
    \definecolor{linkcolor}{rgb}{.71,0.21,0.01}
    \definecolor{citecolor}{rgb}{.12,.54,.11}

    % ANSI colors
    \definecolor{ansi-black}{HTML}{3E424D}
    \definecolor{ansi-black-intense}{HTML}{282C36}
    \definecolor{ansi-red}{HTML}{E75C58}
    \definecolor{ansi-red-intense}{HTML}{B22B31}
    \definecolor{ansi-green}{HTML}{00A250}
    \definecolor{ansi-green-intense}{HTML}{007427}
    \definecolor{ansi-yellow}{HTML}{DDB62B}
    \definecolor{ansi-yellow-intense}{HTML}{B27D12}
    \definecolor{ansi-blue}{HTML}{208FFB}
    \definecolor{ansi-blue-intense}{HTML}{0065CA}
    \definecolor{ansi-magenta}{HTML}{D160C4}
    \definecolor{ansi-magenta-intense}{HTML}{A03196}
    \definecolor{ansi-cyan}{HTML}{60C6C8}
    \definecolor{ansi-cyan-intense}{HTML}{258F8F}
    \definecolor{ansi-white}{HTML}{C5C1B4}
    \definecolor{ansi-white-intense}{HTML}{A1A6B2}
    \definecolor{ansi-default-inverse-fg}{HTML}{FFFFFF}
    \definecolor{ansi-default-inverse-bg}{HTML}{000000}

    % commands and environments needed by pandoc snippets
    % extracted from the output of `pandoc -s`
    \providecommand{\tightlist}{%
      \setlength{\itemsep}{0pt}\setlength{\parskip}{0pt}}
    \DefineVerbatimEnvironment{Highlighting}{Verbatim}{commandchars=\\\{\}}
    % Add ',fontsize=\small' for more characters per line
    \newenvironment{Shaded}{}{}
    \newcommand{\KeywordTok}[1]{\textcolor[rgb]{0.00,0.44,0.13}{\textbf{{#1}}}}
    \newcommand{\DataTypeTok}[1]{\textcolor[rgb]{0.56,0.13,0.00}{{#1}}}
    \newcommand{\DecValTok}[1]{\textcolor[rgb]{0.25,0.63,0.44}{{#1}}}
    \newcommand{\BaseNTok}[1]{\textcolor[rgb]{0.25,0.63,0.44}{{#1}}}
    \newcommand{\FloatTok}[1]{\textcolor[rgb]{0.25,0.63,0.44}{{#1}}}
    \newcommand{\CharTok}[1]{\textcolor[rgb]{0.25,0.44,0.63}{{#1}}}
    \newcommand{\StringTok}[1]{\textcolor[rgb]{0.25,0.44,0.63}{{#1}}}
    \newcommand{\CommentTok}[1]{\textcolor[rgb]{0.38,0.63,0.69}{\textit{{#1}}}}
    \newcommand{\OtherTok}[1]{\textcolor[rgb]{0.00,0.44,0.13}{{#1}}}
    \newcommand{\AlertTok}[1]{\textcolor[rgb]{1.00,0.00,0.00}{\textbf{{#1}}}}
    \newcommand{\FunctionTok}[1]{\textcolor[rgb]{0.02,0.16,0.49}{{#1}}}
    \newcommand{\RegionMarkerTok}[1]{{#1}}
    \newcommand{\ErrorTok}[1]{\textcolor[rgb]{1.00,0.00,0.00}{\textbf{{#1}}}}
    \newcommand{\NormalTok}[1]{{#1}}
    
    % Additional commands for more recent versions of Pandoc
    \newcommand{\ConstantTok}[1]{\textcolor[rgb]{0.53,0.00,0.00}{{#1}}}
    \newcommand{\SpecialCharTok}[1]{\textcolor[rgb]{0.25,0.44,0.63}{{#1}}}
    \newcommand{\VerbatimStringTok}[1]{\textcolor[rgb]{0.25,0.44,0.63}{{#1}}}
    \newcommand{\SpecialStringTok}[1]{\textcolor[rgb]{0.73,0.40,0.53}{{#1}}}
    \newcommand{\ImportTok}[1]{{#1}}
    \newcommand{\DocumentationTok}[1]{\textcolor[rgb]{0.73,0.13,0.13}{\textit{{#1}}}}
    \newcommand{\AnnotationTok}[1]{\textcolor[rgb]{0.38,0.63,0.69}{\textbf{\textit{{#1}}}}}
    \newcommand{\CommentVarTok}[1]{\textcolor[rgb]{0.38,0.63,0.69}{\textbf{\textit{{#1}}}}}
    \newcommand{\VariableTok}[1]{\textcolor[rgb]{0.10,0.09,0.49}{{#1}}}
    \newcommand{\ControlFlowTok}[1]{\textcolor[rgb]{0.00,0.44,0.13}{\textbf{{#1}}}}
    \newcommand{\OperatorTok}[1]{\textcolor[rgb]{0.40,0.40,0.40}{{#1}}}
    \newcommand{\BuiltInTok}[1]{{#1}}
    \newcommand{\ExtensionTok}[1]{{#1}}
    \newcommand{\PreprocessorTok}[1]{\textcolor[rgb]{0.74,0.48,0.00}{{#1}}}
    \newcommand{\AttributeTok}[1]{\textcolor[rgb]{0.49,0.56,0.16}{{#1}}}
    \newcommand{\InformationTok}[1]{\textcolor[rgb]{0.38,0.63,0.69}{\textbf{\textit{{#1}}}}}
    \newcommand{\WarningTok}[1]{\textcolor[rgb]{0.38,0.63,0.69}{\textbf{\textit{{#1}}}}}
    
    
    % Define a nice break command that doesn't care if a line doesn't already
    % exist.
    \def\br{\hspace*{\fill} \\* }
    % Math Jax compatibility definitions
    \def\gt{>}
    \def\lt{<}
    \let\Oldtex\TeX
    \let\Oldlatex\LaTeX
    \renewcommand{\TeX}{\textrm{\Oldtex}}
    \renewcommand{\LaTeX}{\textrm{\Oldlatex}}
    % Document parameters
    % Document title
    \title{Assignment 7}
    \author{Jiarong Ye}
    
    
    
    
    

    % Pygments definitions
    
\makeatletter
\def\PY@reset{\let\PY@it=\relax \let\PY@bf=\relax%
    \let\PY@ul=\relax \let\PY@tc=\relax%
    \let\PY@bc=\relax \let\PY@ff=\relax}
\def\PY@tok#1{\csname PY@tok@#1\endcsname}
\def\PY@toks#1+{\ifx\relax#1\empty\else%
    \PY@tok{#1}\expandafter\PY@toks\fi}
\def\PY@do#1{\PY@bc{\PY@tc{\PY@ul{%
    \PY@it{\PY@bf{\PY@ff{#1}}}}}}}
\def\PY#1#2{\PY@reset\PY@toks#1+\relax+\PY@do{#2}}

\expandafter\def\csname PY@tok@w\endcsname{\def\PY@tc##1{\textcolor[rgb]{0.73,0.73,0.73}{##1}}}
\expandafter\def\csname PY@tok@c\endcsname{\let\PY@it=\textit\def\PY@tc##1{\textcolor[rgb]{0.25,0.50,0.50}{##1}}}
\expandafter\def\csname PY@tok@cp\endcsname{\def\PY@tc##1{\textcolor[rgb]{0.74,0.48,0.00}{##1}}}
\expandafter\def\csname PY@tok@k\endcsname{\let\PY@bf=\textbf\def\PY@tc##1{\textcolor[rgb]{0.00,0.50,0.00}{##1}}}
\expandafter\def\csname PY@tok@kp\endcsname{\def\PY@tc##1{\textcolor[rgb]{0.00,0.50,0.00}{##1}}}
\expandafter\def\csname PY@tok@kt\endcsname{\def\PY@tc##1{\textcolor[rgb]{0.69,0.00,0.25}{##1}}}
\expandafter\def\csname PY@tok@o\endcsname{\def\PY@tc##1{\textcolor[rgb]{0.40,0.40,0.40}{##1}}}
\expandafter\def\csname PY@tok@ow\endcsname{\let\PY@bf=\textbf\def\PY@tc##1{\textcolor[rgb]{0.67,0.13,1.00}{##1}}}
\expandafter\def\csname PY@tok@nb\endcsname{\def\PY@tc##1{\textcolor[rgb]{0.00,0.50,0.00}{##1}}}
\expandafter\def\csname PY@tok@nf\endcsname{\def\PY@tc##1{\textcolor[rgb]{0.00,0.00,1.00}{##1}}}
\expandafter\def\csname PY@tok@nc\endcsname{\let\PY@bf=\textbf\def\PY@tc##1{\textcolor[rgb]{0.00,0.00,1.00}{##1}}}
\expandafter\def\csname PY@tok@nn\endcsname{\let\PY@bf=\textbf\def\PY@tc##1{\textcolor[rgb]{0.00,0.00,1.00}{##1}}}
\expandafter\def\csname PY@tok@ne\endcsname{\let\PY@bf=\textbf\def\PY@tc##1{\textcolor[rgb]{0.82,0.25,0.23}{##1}}}
\expandafter\def\csname PY@tok@nv\endcsname{\def\PY@tc##1{\textcolor[rgb]{0.10,0.09,0.49}{##1}}}
\expandafter\def\csname PY@tok@no\endcsname{\def\PY@tc##1{\textcolor[rgb]{0.53,0.00,0.00}{##1}}}
\expandafter\def\csname PY@tok@nl\endcsname{\def\PY@tc##1{\textcolor[rgb]{0.63,0.63,0.00}{##1}}}
\expandafter\def\csname PY@tok@ni\endcsname{\let\PY@bf=\textbf\def\PY@tc##1{\textcolor[rgb]{0.60,0.60,0.60}{##1}}}
\expandafter\def\csname PY@tok@na\endcsname{\def\PY@tc##1{\textcolor[rgb]{0.49,0.56,0.16}{##1}}}
\expandafter\def\csname PY@tok@nt\endcsname{\let\PY@bf=\textbf\def\PY@tc##1{\textcolor[rgb]{0.00,0.50,0.00}{##1}}}
\expandafter\def\csname PY@tok@nd\endcsname{\def\PY@tc##1{\textcolor[rgb]{0.67,0.13,1.00}{##1}}}
\expandafter\def\csname PY@tok@s\endcsname{\def\PY@tc##1{\textcolor[rgb]{0.73,0.13,0.13}{##1}}}
\expandafter\def\csname PY@tok@sd\endcsname{\let\PY@it=\textit\def\PY@tc##1{\textcolor[rgb]{0.73,0.13,0.13}{##1}}}
\expandafter\def\csname PY@tok@si\endcsname{\let\PY@bf=\textbf\def\PY@tc##1{\textcolor[rgb]{0.73,0.40,0.53}{##1}}}
\expandafter\def\csname PY@tok@se\endcsname{\let\PY@bf=\textbf\def\PY@tc##1{\textcolor[rgb]{0.73,0.40,0.13}{##1}}}
\expandafter\def\csname PY@tok@sr\endcsname{\def\PY@tc##1{\textcolor[rgb]{0.73,0.40,0.53}{##1}}}
\expandafter\def\csname PY@tok@ss\endcsname{\def\PY@tc##1{\textcolor[rgb]{0.10,0.09,0.49}{##1}}}
\expandafter\def\csname PY@tok@sx\endcsname{\def\PY@tc##1{\textcolor[rgb]{0.00,0.50,0.00}{##1}}}
\expandafter\def\csname PY@tok@m\endcsname{\def\PY@tc##1{\textcolor[rgb]{0.40,0.40,0.40}{##1}}}
\expandafter\def\csname PY@tok@gh\endcsname{\let\PY@bf=\textbf\def\PY@tc##1{\textcolor[rgb]{0.00,0.00,0.50}{##1}}}
\expandafter\def\csname PY@tok@gu\endcsname{\let\PY@bf=\textbf\def\PY@tc##1{\textcolor[rgb]{0.50,0.00,0.50}{##1}}}
\expandafter\def\csname PY@tok@gd\endcsname{\def\PY@tc##1{\textcolor[rgb]{0.63,0.00,0.00}{##1}}}
\expandafter\def\csname PY@tok@gi\endcsname{\def\PY@tc##1{\textcolor[rgb]{0.00,0.63,0.00}{##1}}}
\expandafter\def\csname PY@tok@gr\endcsname{\def\PY@tc##1{\textcolor[rgb]{1.00,0.00,0.00}{##1}}}
\expandafter\def\csname PY@tok@ge\endcsname{\let\PY@it=\textit}
\expandafter\def\csname PY@tok@gs\endcsname{\let\PY@bf=\textbf}
\expandafter\def\csname PY@tok@gp\endcsname{\let\PY@bf=\textbf\def\PY@tc##1{\textcolor[rgb]{0.00,0.00,0.50}{##1}}}
\expandafter\def\csname PY@tok@go\endcsname{\def\PY@tc##1{\textcolor[rgb]{0.53,0.53,0.53}{##1}}}
\expandafter\def\csname PY@tok@gt\endcsname{\def\PY@tc##1{\textcolor[rgb]{0.00,0.27,0.87}{##1}}}
\expandafter\def\csname PY@tok@err\endcsname{\def\PY@bc##1{\setlength{\fboxsep}{0pt}\fcolorbox[rgb]{1.00,0.00,0.00}{1,1,1}{\strut ##1}}}
\expandafter\def\csname PY@tok@kc\endcsname{\let\PY@bf=\textbf\def\PY@tc##1{\textcolor[rgb]{0.00,0.50,0.00}{##1}}}
\expandafter\def\csname PY@tok@kd\endcsname{\let\PY@bf=\textbf\def\PY@tc##1{\textcolor[rgb]{0.00,0.50,0.00}{##1}}}
\expandafter\def\csname PY@tok@kn\endcsname{\let\PY@bf=\textbf\def\PY@tc##1{\textcolor[rgb]{0.00,0.50,0.00}{##1}}}
\expandafter\def\csname PY@tok@kr\endcsname{\let\PY@bf=\textbf\def\PY@tc##1{\textcolor[rgb]{0.00,0.50,0.00}{##1}}}
\expandafter\def\csname PY@tok@bp\endcsname{\def\PY@tc##1{\textcolor[rgb]{0.00,0.50,0.00}{##1}}}
\expandafter\def\csname PY@tok@fm\endcsname{\def\PY@tc##1{\textcolor[rgb]{0.00,0.00,1.00}{##1}}}
\expandafter\def\csname PY@tok@vc\endcsname{\def\PY@tc##1{\textcolor[rgb]{0.10,0.09,0.49}{##1}}}
\expandafter\def\csname PY@tok@vg\endcsname{\def\PY@tc##1{\textcolor[rgb]{0.10,0.09,0.49}{##1}}}
\expandafter\def\csname PY@tok@vi\endcsname{\def\PY@tc##1{\textcolor[rgb]{0.10,0.09,0.49}{##1}}}
\expandafter\def\csname PY@tok@vm\endcsname{\def\PY@tc##1{\textcolor[rgb]{0.10,0.09,0.49}{##1}}}
\expandafter\def\csname PY@tok@sa\endcsname{\def\PY@tc##1{\textcolor[rgb]{0.73,0.13,0.13}{##1}}}
\expandafter\def\csname PY@tok@sb\endcsname{\def\PY@tc##1{\textcolor[rgb]{0.73,0.13,0.13}{##1}}}
\expandafter\def\csname PY@tok@sc\endcsname{\def\PY@tc##1{\textcolor[rgb]{0.73,0.13,0.13}{##1}}}
\expandafter\def\csname PY@tok@dl\endcsname{\def\PY@tc##1{\textcolor[rgb]{0.73,0.13,0.13}{##1}}}
\expandafter\def\csname PY@tok@s2\endcsname{\def\PY@tc##1{\textcolor[rgb]{0.73,0.13,0.13}{##1}}}
\expandafter\def\csname PY@tok@sh\endcsname{\def\PY@tc##1{\textcolor[rgb]{0.73,0.13,0.13}{##1}}}
\expandafter\def\csname PY@tok@s1\endcsname{\def\PY@tc##1{\textcolor[rgb]{0.73,0.13,0.13}{##1}}}
\expandafter\def\csname PY@tok@mb\endcsname{\def\PY@tc##1{\textcolor[rgb]{0.40,0.40,0.40}{##1}}}
\expandafter\def\csname PY@tok@mf\endcsname{\def\PY@tc##1{\textcolor[rgb]{0.40,0.40,0.40}{##1}}}
\expandafter\def\csname PY@tok@mh\endcsname{\def\PY@tc##1{\textcolor[rgb]{0.40,0.40,0.40}{##1}}}
\expandafter\def\csname PY@tok@mi\endcsname{\def\PY@tc##1{\textcolor[rgb]{0.40,0.40,0.40}{##1}}}
\expandafter\def\csname PY@tok@il\endcsname{\def\PY@tc##1{\textcolor[rgb]{0.40,0.40,0.40}{##1}}}
\expandafter\def\csname PY@tok@mo\endcsname{\def\PY@tc##1{\textcolor[rgb]{0.40,0.40,0.40}{##1}}}
\expandafter\def\csname PY@tok@ch\endcsname{\let\PY@it=\textit\def\PY@tc##1{\textcolor[rgb]{0.25,0.50,0.50}{##1}}}
\expandafter\def\csname PY@tok@cm\endcsname{\let\PY@it=\textit\def\PY@tc##1{\textcolor[rgb]{0.25,0.50,0.50}{##1}}}
\expandafter\def\csname PY@tok@cpf\endcsname{\let\PY@it=\textit\def\PY@tc##1{\textcolor[rgb]{0.25,0.50,0.50}{##1}}}
\expandafter\def\csname PY@tok@c1\endcsname{\let\PY@it=\textit\def\PY@tc##1{\textcolor[rgb]{0.25,0.50,0.50}{##1}}}
\expandafter\def\csname PY@tok@cs\endcsname{\let\PY@it=\textit\def\PY@tc##1{\textcolor[rgb]{0.25,0.50,0.50}{##1}}}

\def\PYZbs{\char`\\}
\def\PYZus{\char`\_}
\def\PYZob{\char`\{}
\def\PYZcb{\char`\}}
\def\PYZca{\char`\^}
\def\PYZam{\char`\&}
\def\PYZlt{\char`\<}
\def\PYZgt{\char`\>}
\def\PYZsh{\char`\#}
\def\PYZpc{\char`\%}
\def\PYZdl{\char`\$}
\def\PYZhy{\char`\-}
\def\PYZsq{\char`\'}
\def\PYZdq{\char`\"}
\def\PYZti{\char`\~}
% for compatibility with earlier versions
\def\PYZat{@}
\def\PYZlb{[}
\def\PYZrb{]}
\makeatother


    % Exact colors from NB
    \definecolor{incolor}{rgb}{0.0, 0.0, 0.5}
    \definecolor{outcolor}{rgb}{0.545, 0.0, 0.0}



    
    % Prevent overflowing lines due to hard-to-break entities
    \sloppy 
    % Setup hyperref package
    \hypersetup{
      breaklinks=true,  % so long urls are correctly broken across lines
      colorlinks=true,
      urlcolor=urlcolor,
      linkcolor=linkcolor,
      citecolor=citecolor,
      }
    % Slightly bigger margins than the latex defaults
    
    \geometry{verbose,tmargin=1in,bmargin=1in,lmargin=1in,rmargin=1in}
    
    

    \begin{document}
    
    
    \maketitle
    
    

    
    Your homework will consider an experiment on battery life for different
types and brands of battery. Two brands (a name brand and a generic
brand) of two types (Alkaline and ``Heavy Duty'') of batteries were
tested to see how long they could run continuously. This results in four
categories,

\begin{itemize}
\tightlist
\item
  AlkName is for name-brand alkaline batteries,
\item
  AlkGen is for generic alkaline batteries,
\item
  HDName is for heavy duty name-brand batteries,
\item
  HDGen is for generic heavy duty batteries.
\end{itemize}

Four batteries of each type were tested and the times to battery failure
are recorded as below. Use the code below to read in the data:

    \subsubsection*{Read Data}\label{read-data}

    \begin{Verbatim}[commandchars=\\\{\}]
{\color{incolor}In [{\color{incolor}1}]:} type\PY{o}{=}\PY{k+kt}{c}\PY{p}{(}\PY{l+s}{\PYZdq{}}\PY{l+s}{AlkName\PYZdq{}}\PY{p}{,}\PY{l+s}{\PYZdq{}}\PY{l+s}{AlkName\PYZdq{}}\PY{p}{,}\PY{l+s}{\PYZdq{}}\PY{l+s}{AlkName\PYZdq{}}\PY{p}{,}\PY{l+s}{\PYZdq{}}\PY{l+s}{AlkName\PYZdq{}}\PY{p}{,}\PY{l+s}{\PYZdq{}}\PY{l+s}{AlkGen\PYZdq{}}\PY{p}{,}\PY{l+s}{\PYZdq{}}\PY{l+s}{AlkGen\PYZdq{}}\PY{p}{,}\PY{l+s}{\PYZdq{}}\PY{l+s}{AlkGen\PYZdq{}}\PY{p}{,}\PY{l+s}{\PYZdq{}}\PY{l+s}{AlkGen\PYZdq{}}\PY{p}{,}
        \PY{l+s}{\PYZdq{}}\PY{l+s}{HDName\PYZdq{}}\PY{p}{,}\PY{l+s}{\PYZdq{}}\PY{l+s}{HDName\PYZdq{}}\PY{p}{,}\PY{l+s}{\PYZdq{}}\PY{l+s}{HDName\PYZdq{}}\PY{p}{,}\PY{l+s}{\PYZdq{}}\PY{l+s}{HDName\PYZdq{}}\PY{p}{,}\PY{l+s}{\PYZdq{}}\PY{l+s}{HDGen\PYZdq{}}\PY{p}{,}\PY{l+s}{\PYZdq{}}\PY{l+s}{HDGen\PYZdq{}}\PY{p}{,}\PY{l+s}{\PYZdq{}}\PY{l+s}{HDGen\PYZdq{}}\PY{p}{,}\PY{l+s}{\PYZdq{}}\PY{l+s}{HDGen\PYZdq{}}\PY{p}{)}
        life\PY{o}{=}\PY{k+kt}{c}\PY{p}{(}\PY{l+m}{100.668}\PY{p}{,} \PY{l+m}{77.734}\PY{p}{,}\PY{l+m}{79.210}\PY{p}{,}\PY{l+m}{95.063}\PY{p}{,}\PY{l+m}{206.880}\PY{p}{,}\PY{l+m}{153.347}\PY{p}{,}\PY{l+m}{165.980}\PY{p}{,}\PY{l+m}{196.000}\PY{p}{,}
        \PY{l+m}{14.951}\PY{p}{,}\PY{l+m}{18.063}\PY{p}{,}\PY{l+m}{11.111}\PY{p}{,}\PY{l+m}{12.840}\PY{p}{,}\PY{l+m}{15.340}\PY{p}{,}\PY{l+m}{22.090}\PY{p}{,}\PY{l+m}{15.734}\PY{p}{,} \PY{l+m}{14.440}\PY{p}{)}
        batt\PY{o}{=}\PY{k+kt}{data.frame}\PY{p}{(}type\PY{o}{=}type\PY{p}{,} life\PY{o}{=}life\PY{p}{)}
        batt
\end{Verbatim}

    \begin{tabular}{r|ll}
 type & life\\
\hline
	 AlkName & 100.668\\
	 AlkName &  77.734\\
	 AlkName &  79.210\\
	 AlkName &  95.063\\
	 AlkGen  & 206.880\\
	 AlkGen  & 153.347\\
	 AlkGen  & 165.980\\
	 AlkGen  & 196.000\\
	 HDName  &  14.951\\
	 HDName  &  18.063\\
	 HDName  &  11.111\\
	 HDName  &  12.840\\
	 HDGen   &  15.340\\
	 HDGen   &  22.090\\
	 HDGen   &  15.734\\
	 HDGen   &  14.440\\
\end{tabular}


    
    \subsubsection*{Q1}\label{q1}

Plot the data

    \begin{Verbatim}[commandchars=\\\{\}]
{\color{incolor}In [{\color{incolor}4}]:} \PY{k+kn}{library}\PY{p}{(}ggplot2\PY{p}{)}
        ggplot\PY{p}{(}batt\PY{p}{,} aes\PY{p}{(}x\PY{o}{=}type\PY{p}{,} y\PY{o}{=}life\PY{p}{,} color\PY{o}{=}type\PY{p}{)}\PY{p}{)} \PY{o}{+}
                geom\PYZus{}boxplot\PY{p}{(}\PY{p}{)} \PY{o}{+}
                ylab\PY{p}{(}\PY{l+s}{\PYZsq{}}\PY{l+s}{life\PYZsq{}}\PY{p}{)} \PY{o}{+}
                ggtitle\PY{p}{(}\PY{l+s}{\PYZsq{}}\PY{l+s}{Boxplots of battery life of different battery types\PYZsq{}}\PY{p}{)} \PY{o}{+}
                theme\PY{p}{(}plot.title \PY{o}{=} element\PYZus{}text\PY{p}{(}hjust \PY{o}{=} \PY{l+m}{0.5}\PY{p}{)}\PY{p}{)}
\end{Verbatim}

    
    
    \begin{center}
    \adjustimage{max size={0.7\linewidth}{0.7\paperheight}}{output_4_1.png}
    \end{center}
    { \hspace*{\fill} \\}
    
    \subsubsection*{Q2}\label{q2}

For the battery data, do the following:

\begin{itemize}
\item
  \begin{enumerate}
  \def\labelenumi{(\alph{enumi})}
  \tightlist
  \item
    Write out the one-way ANOVA model for this data.
  \end{enumerate}
\item
  \begin{enumerate}
  \def\labelenumi{(\alph{enumi})}
  \setcounter{enumi}{1}
  \tightlist
  \item
    Show residual plots for this model. Are the residuals approximately
    normal? Justify your answer.
  \end{enumerate}
\item
  \begin{enumerate}
  \def\labelenumi{(\alph{enumi})}
  \setcounter{enumi}{2}
  \tightlist
  \item
    Is the assumption of constant error variance among treatments
    justified? Explain your answer.
  \end{enumerate}
\end{itemize}

    \paragraph{(a)}\label{a}

    \begin{Verbatim}[commandchars=\\\{\}]
{\color{incolor}In [{\color{incolor}7}]:} \PY{k+kn}{library}\PY{p}{(}knitr\PY{p}{)}
        \PY{k+kn}{library}\PY{p}{(}lsmeans\PY{p}{)}
        aov.batt \PY{o}{=} aov\PY{p}{(}life\PY{o}{\PYZti{}}type\PY{p}{,} data \PY{o}{=} batt\PY{p}{)}
        kable\PY{p}{(}anova\PY{p}{(}aov.batt\PY{p}{)}\PY{p}{,} format\PY{o}{=}\PY{l+s}{\PYZsq{}}\PY{l+s}{markdown\PYZsq{}}\PY{p}{)}
\end{Verbatim}

    
    \begin{verbatim}


|          | Df|    Sum Sq|    Mean Sq| F value| Pr(>F)|
|:---------|--:|---------:|----------:|-------:|------:|
|type      |  3| 73526.811| 24508.9369| 125.643|      0|
|Residuals | 12|  2340.817|   195.0681|      NA|     NA|
    \end{verbatim}

    
    \paragraph{(b)}\label{b}

    \begin{Verbatim}[commandchars=\\\{\}]
{\color{incolor}In [{\color{incolor}9}]:} par\PY{p}{(}mfrow\PY{o}{=}\PY{k+kt}{c}\PY{p}{(}\PY{l+m}{2}\PY{p}{,}\PY{l+m}{2}\PY{p}{)}\PY{p}{)}
        plot\PY{p}{(}aov.batt\PY{p}{)}
\end{Verbatim}

    \begin{center}
    \adjustimage{max size={0.7\linewidth}{0.7\paperheight}}{output_9_0.png}
    \end{center}
    { \hspace*{\fill} \\}
    
    From the QQ-plot above we could conclude that since not all the points
fall on the dotted line, thus the residuals are not normal, it also
appears to be heavy tailed.

    \paragraph{(c)}\label{c}

    From the Residual vs. Fitted plot we can see that for each vertical line
of points representing a different treatment, the spread on the points
does not appear to be equal. It presents a trumpet pattern, indicating
that these 3 treatments do not have the same variance, the 3rd treatment
has vastly larger spread than the others. So the assumption of constant
variance is violated.

    \subsubsection*{Q3}\label{q3}

Now consider using the square-root of the battery life as a response
variable. Repeat (a)-(c) above for this transformation.

    \paragraph{(a)}\label{a}

    \begin{Verbatim}[commandchars=\\\{\}]
{\color{incolor}In [{\color{incolor}12}]:} batt\PY{o}{\PYZdl{}}sqrt\PYZus{}life \PY{o}{=} \PY{k+kp}{sqrt}\PY{p}{(}batt\PY{o}{\PYZdl{}}life\PY{p}{)}
         aov.batt \PY{o}{=} aov\PY{p}{(}sqrt\PYZus{}life\PY{o}{\PYZti{}}type\PY{p}{,} data \PY{o}{=} batt\PY{p}{)}
         kable\PY{p}{(}anova\PY{p}{(}aov.batt\PY{p}{)}\PY{p}{,} format\PY{o}{=}\PY{l+s}{\PYZsq{}}\PY{l+s}{markdown\PYZsq{}}\PY{p}{)}
\end{Verbatim}

    
    \begin{verbatim}


|          | Df|     Sum Sq|    Mean Sq|  F value| Pr(>F)|
|:---------|--:|----------:|----------:|--------:|------:|
|type      |  3| 255.835800| 85.2785998| 217.5284|      0|
|Residuals | 12|   4.704412|  0.3920344|       NA|     NA|
    \end{verbatim}

    
    \paragraph{(b)}\label{b}

    \begin{Verbatim}[commandchars=\\\{\}]
{\color{incolor}In [{\color{incolor}13}]:} par\PY{p}{(}mfrow\PY{o}{=}\PY{k+kt}{c}\PY{p}{(}\PY{l+m}{2}\PY{p}{,}\PY{l+m}{2}\PY{p}{)}\PY{p}{)}
         plot\PY{p}{(}aov.batt\PY{p}{)}
\end{Verbatim}

    \begin{center}
    \adjustimage{max size={0.7\linewidth}{0.7\paperheight}}{output_17_0.png}
    \end{center}
    { \hspace*{\fill} \\}
    
    From the QQ-plot above we could conclude that although most of  the points
fall on the dotted line, except for a few on the top right and lower left of the plot, thus the residuals are not  normal.

    \paragraph{(c)}\label{c}

    From the Residual vs. Fitted plot we can see that for each vertical line
of points representing a different treatment, the spread on the points
seems more evenly distributed around the horizontal baseline than the
result before transformation, but the 3rd treatment still appears to be more spread-out than the other two, indicating that these 3 treatments do not have
the same variance. So the assumption of constant variance is still
violated.

    \subsubsection*{Q4}\label{q4}

Now consider using the log of the battery life as a response variable.
Repeat (a)-(c) above for this transformation.

    \paragraph{(a)}\label{a}

    \begin{Verbatim}[commandchars=\\\{\}]
{\color{incolor}In [{\color{incolor}14}]:} batt\PY{o}{\PYZdl{}}log\PYZus{}life \PY{o}{=} \PY{k+kp}{log}\PY{p}{(}batt\PY{o}{\PYZdl{}}life\PY{p}{)}
         aov.batt \PY{o}{=} aov\PY{p}{(}log\PYZus{}life\PY{o}{\PYZti{}}type\PY{p}{,} data \PY{o}{=} batt\PY{p}{)}
         kable\PY{p}{(}anova\PY{p}{(}aov.batt\PY{p}{)}\PY{p}{,} format\PY{o}{=}\PY{l+s}{\PYZsq{}}\PY{l+s}{markdown\PYZsq{}}\PY{p}{)}
\end{Verbatim}

    
    \begin{verbatim}


|          | Df|     Sum Sq|   Mean Sq|  F value| Pr(>F)|
|:---------|--:|----------:|---------:|--------:|------:|
|type      |  3| 18.8002838| 6.2667613| 215.1637|      0|
|Residuals | 12|  0.3495066| 0.0291256|       NA|     NA|
    \end{verbatim}

    
    \paragraph{(b)}\label{b}

    \begin{Verbatim}[commandchars=\\\{\}]
{\color{incolor}In [{\color{incolor}15}]:} par\PY{p}{(}mfrow\PY{o}{=}\PY{k+kt}{c}\PY{p}{(}\PY{l+m}{2}\PY{p}{,}\PY{l+m}{2}\PY{p}{)}\PY{p}{)}
         plot\PY{p}{(}aov.batt\PY{p}{)}
\end{Verbatim}

    \begin{center}
    \adjustimage{max size={0.7\linewidth}{0.7\paperheight}}{output_25_0.png}
    \end{center}
    { \hspace*{\fill} \\}
    
    From the QQ-plot above we could conclude that basically all the points
fall on the dotted line, thus the residuals are approximately normal.

    \paragraph{(c)}\label{c}

    From the Residual vs. Fitted plot we can see that for each vertical line
of points representing a different treatment, the spread on the points appears to be approximately equal, indicating
that these 3 treatments have the same variance. So the assumption of
constant variance is not violated.

    \subsubsection*{Q5}\label{q5}

Now consider using the square of the battery life as a response
variable. Repeat (a)-(c) above for this transformation.

    \paragraph{(a)}\label{a}

    \begin{Verbatim}[commandchars=\\\{\}]
{\color{incolor}In [{\color{incolor}20}]:} batt\PY{o}{\PYZdl{}}square\PYZus{}life \PY{o}{=} batt\PY{o}{\PYZdl{}}life\PY{o}{\PYZca{}}\PY{l+m}{2}
         aov.batt \PY{o}{=} aov\PY{p}{(}square\PYZus{}life\PY{o}{\PYZti{}}type\PY{p}{,} data \PY{o}{=} batt\PY{p}{)}
         kable\PY{p}{(}anova\PY{p}{(}aov.batt\PY{p}{)}\PY{p}{,} format\PY{o}{=}\PY{l+s}{\PYZsq{}}\PY{l+s}{markdown\PYZsq{}}\PY{p}{)}
\end{Verbatim}

    
    \begin{verbatim}


|          | Df|     Sum Sq|   Mean Sq|  F value| Pr(>F)|
|:---------|--:|----------:|---------:|--------:|------:|
|type      |  3| 2905095650| 968365217| 45.13527|  8e-07|
|Residuals | 12|  257456832|  21454736|       NA|     NA|
    \end{verbatim}

    
    \paragraph{(b)}\label{b}

    \begin{Verbatim}[commandchars=\\\{\}]
{\color{incolor}In [{\color{incolor}21}]:} par\PY{p}{(}mfrow\PY{o}{=}\PY{k+kt}{c}\PY{p}{(}\PY{l+m}{2}\PY{p}{,}\PY{l+m}{2}\PY{p}{)}\PY{p}{)}
         plot\PY{p}{(}aov.batt\PY{p}{)}
\end{Verbatim}

    \begin{center}
    \adjustimage{max size={0.7\linewidth}{0.7\paperheight}}{output_33_0.png}
    \end{center}
    { \hspace*{\fill} \\}
    
    From the QQ-plot above we could conclude that since not all the points
fall on the dotted line, thus the residuals are not normal, it also
appears to be VERY heavy tailed.

    \paragraph{(c)}\label{c}

    From the Residual vs. Fitted plot we can see that for each vertical line
of points representing a different treatment, the spread on the points
does not appear to be equal. It presents a trumpet pattern, indicating
that these 3 treatments do not have the same variance, the 3rd treatment
has vastly larger spread than the others. So the assumption of constant
variance is violated.

    \subsubsection*{Q6}\label{q6}

Which of the four models you have fit has residuals that best satisfy
the assumptions of the ANOVA model? Explain your choice.\\

\noindent    \textbf{The log transformation appears to be the best for
satisfying the normality and constant variance assumptions of residuals}

    \begin{Verbatim}[commandchars=\\\{\}]
{\color{incolor}In [{\color{incolor}24}]:} batt\PY{o}{\PYZdl{}}log\PYZus{}life \PY{o}{=} \PY{k+kp}{log}\PY{p}{(}batt\PY{o}{\PYZdl{}}life\PY{p}{)}
         aov.batt \PY{o}{=} aov\PY{p}{(}log\PYZus{}life\PY{o}{\PYZti{}}type\PY{p}{,} data \PY{o}{=} batt\PY{p}{)}
         kable\PY{p}{(}anova\PY{p}{(}aov.batt\PY{p}{)}\PY{p}{,} format\PY{o}{=}\PY{l+s}{\PYZsq{}}\PY{l+s}{markdown\PYZsq{}}\PY{p}{)}
         par\PY{p}{(}mfrow\PY{o}{=}\PY{k+kt}{c}\PY{p}{(}\PY{l+m}{2}\PY{p}{,}\PY{l+m}{2}\PY{p}{)}\PY{p}{)}
         plot\PY{p}{(}aov.batt\PY{p}{)}
\end{Verbatim}

    
    \begin{verbatim}
   
   
   |          | Df|     Sum Sq|   Mean Sq|  F value| Pr(>F)|
   |:---------|--:|----------:|---------:|--------:|------:|
   |type      |  3| 18.8002838| 6.2667613| 215.1637|      0|
   |Residuals | 12|  0.3495066| 0.0291256|       NA|     NA|
   \end{verbatim}

    
    \begin{center}
    \adjustimage{max size={0.7\linewidth}{0.7\paperheight}}{output_25_0.png}
    \end{center}
    { \hspace*{\fill} \\}
    
    \begin{itemize}
\item
  From the QQ-plot above we could conclude that basically all the points
  fall on the dotted line, thus the residuals are approximately normal
\item
  From the Residual vs. Fitted plot we can see that for each vertical
  line of points representing a different treatment, the spread on the
  points seems more evenly distributed around the horizontal baseline
  than the result before transformation, indicating that these 3
  treatments have the same variance. So the assumption of constant
  variance is not violated.
\end{itemize}

    \subsubsection*{Q7}\label{q7}

For the model you chose in question 6 above, are there any significant
pairwise differences in mean lifetime of different battery types? If so,
state which are different, and provide p-values, test statistics, and
null hypotheses for the hypothesis tests used.\\\

    Denote log battery life as \(l\),

\begin{itemize}
\tightlist
\item
  Null hypothesis:
\end{itemize}

\[\text{For } Y_{it}=\mu +\tau_i+\epsilon_{it},  \text{where } i=AlkGen,AlkName,HDGen,HDName\]

\[H_0: \tau _i= 0 \text{ for } i=AlkGen,AlkName,HDGen,HDName\]


\begin{Verbatim}[commandchars=\\\{\}]
{\color{incolor}In [{\color{incolor}48}]:} aov.batt \PY{o}{=} aov\PY{p}{(}log\PYZus{}life\PY{o}{\PYZti{}}type\PY{p}{,} data \PY{o}{=} batt\PY{p}{)}
lsm.batt\PY{o}{=}lsmeans\PY{p}{(}aov.batt\PY{p}{,} \PY{o}{\PYZti{}} type\PY{p}{)}
\PY{k+kp}{summary}\PY{p}{(}contrast\PY{p}{(}lsm.batt\PY{p}{,} method\PY{o}{=}\PY{l+s}{\PYZdq{}}\PY{l+s}{pairwise\PYZdq{}}\PY{p}{,} adjust\PY{o}{=}\PY{l+s}{\PYZdq{}}\PY{l+s}{tukey\PYZdq{}}\PY{p}{)}\PY{p}{,}
infer\PY{o}{=}\PY{k+kt}{c}\PY{p}{(}\PY{n+nb+bp}{T}\PY{p}{,}\PY{n+nb+bp}{T}\PY{p}{)}\PY{p}{,} level\PY{o}{=}\PY{l+m}{0.95}\PY{p}{,} side\PY{o}{=}\PY{l+s}{\PYZdq{}}\PY{l+s}{two\PYZhy{}sided\PYZdq{}}\PY{p}{)}
\end{Verbatim}

\begin{tabular}{r|llllllll}
	contrast & estimate & SE & df & lower.CL & upper.CL & t.ratio & p.value\\
	\hline
	AlkGen - AlkName & 0.7157653        & 0.1206763        & 12               &  0.3574892       & 1.0740414        &  5.931281        & 3.480708e-04    \\
	AlkGen - HDGen   & 2.3758523        & 0.1206763        & 12               &  2.0175762       & 2.7341284        & 19.687807        & 1.011516e-09    \\
	AlkGen - HDName  & 2.5489199        & 0.1206763        & 12               &  2.1906438       & 2.9071960        & 21.121954        & 4.272094e-10    \\
	AlkName - HDGen  & 1.6600870        & 0.1206763        & 12               &  1.3018109       & 2.0183631        & 13.756526        & 5.448802e-08    \\
	AlkName - HDName & 1.8331546        & 0.1206763        & 12               &  1.4748785       & 2.1914307        & 15.190672        & 1.761964e-08    \\
	HDGen - HDName   & 0.1730675        & 0.1206763        & 12               & -0.1852085       & 0.5313436        &  1.434147        & 5.034253e-01    \\
\end{tabular}



\begin{Verbatim}[commandchars=\\\{\}]
{\color{incolor}In [{\color{incolor}49}]:} AlkGen\PYZus{}AlkName\PYZus{}pv \PY{o}{=} \PY{l+m}{3.480708e\PYZhy{}04}
AlkGen\PYZus{}HDGen\PYZus{}pv \PY{o}{=} \PY{l+m}{1.011516e\PYZhy{}09}
AlkGen\PYZus{}HDName\PYZus{}pv \PY{o}{=} \PY{l+m}{4.272094e\PYZhy{}10}
AlkName\PYZus{}HDGen\PYZus{}pv \PY{o}{=} \PY{l+m}{5.448802e\PYZhy{}08}
AlkName\PYZus{}HDName\PYZus{}pv \PY{o}{=} \PY{l+m}{1.761964e\PYZhy{}08}
HDGen\PYZus{}HDName\PYZus{}pv \PY{o}{=} \PY{l+m}{5.034253e\PYZhy{}01}

AlkGen\PYZus{}AlkName\PYZus{}pv \PY{o}{\PYZlt{}} \PY{l+m}{0.5}
AlkGen\PYZus{}HDGen\PYZus{}pv \PY{o}{\PYZlt{}} \PY{l+m}{0.5}
AlkGen\PYZus{}HDName\PYZus{}pv \PY{o}{\PYZlt{}} \PY{l+m}{0.5}
AlkName\PYZus{}HDGen\PYZus{}pv \PY{o}{\PYZlt{}} \PY{l+m}{0.5}
AlkName\PYZus{}HDName\PYZus{}pv \PY{o}{\PYZlt{}} \PY{l+m}{0.5}
HDGen\PYZus{}HDName\PYZus{}pv \PY{o}{\PYZlt{}} \PY{l+m}{0.5}
\end{Verbatim}

TRUE


TRUE


TRUE


TRUE


TRUE


FALSE

    
    We can interpret the results of these tests with the following
statements:

\begin{itemize}
\item
  \begin{enumerate}
  \def\labelenumi{\arabic{enumi}.}
  \tightlist
  \item
    The battery life of AlkGen is significantly longer than the battery
    life of AlkName;
  \end{enumerate}
\item
  \begin{enumerate}
  \def\labelenumi{\arabic{enumi}.}
  \setcounter{enumi}{1}
  \tightlist
  \item
    The battery life of AlkGen is significantly longer than the battery
    life of HDGen;
  \end{enumerate}
\item
  \begin{enumerate}
  \def\labelenumi{\arabic{enumi}.}
  \setcounter{enumi}{2}
  \tightlist
  \item
    The battery life of AlkGen is significantly longer than the battery
    life of HDName;
  \end{enumerate}
\item
  \begin{enumerate}
  \def\labelenumi{\arabic{enumi}.}
  \setcounter{enumi}{3}
  \tightlist
  \item
    The battery life of AlkName is significantly longer than the battery
    life of HDGen;
  \end{enumerate}
\item
  \begin{enumerate}
  \def\labelenumi{\arabic{enumi}.}
  \setcounter{enumi}{4}
  \tightlist
  \item
    The battery life of AlkName is significantly longer than the battery
    life of HDName;
  \end{enumerate}
\item
  \begin{enumerate}
  \def\labelenumi{\arabic{enumi}.}
  \setcounter{enumi}{2}
  \tightlist
  \item
    The battery life of HDGen is not significantly different from the
    battery life of HDName;
  \end{enumerate}
\end{itemize}

which we could validate with the boxplot:

    \begin{Verbatim}[commandchars=\\\{\}]
{\color{incolor}In [{\color{incolor}32}]:} ggplot\PY{p}{(}batt\PY{p}{,} aes\PY{p}{(}x\PY{o}{=}type\PY{p}{,} y\PY{o}{=}life\PY{p}{,} color\PY{o}{=}type\PY{p}{)}\PY{p}{)} \PY{o}{+}
                 geom\PYZus{}boxplot\PY{p}{(}\PY{p}{)} \PY{o}{+}
                 ylab\PY{p}{(}\PY{l+s}{\PYZsq{}}\PY{l+s}{life\PYZsq{}}\PY{p}{)} \PY{o}{+}
                 ggtitle\PY{p}{(}\PY{l+s}{\PYZsq{}}\PY{l+s}{Boxplots of battery life of different battery types\PYZsq{}}\PY{p}{)} \PY{o}{+}
                 theme\PY{p}{(}plot.title \PY{o}{=} element\PYZus{}text\PY{p}{(}hjust \PY{o}{=} \PY{l+m}{0.5}\PY{p}{)}\PY{p}{)}
\end{Verbatim}

    
    
    \begin{center}
    \adjustimage{max size={0.7\linewidth}{0.7\paperheight}}{output_46_1.png}
    \end{center}
    { \hspace*{\fill} \\}
    
    \subsubsection*{Q8}\label{q8}

Hot Dogs A study was conducted to compare the calories and sodium in hot
dogs made with different types of meat, and plot calories as a response variable with the type of meat on the
x-axis. Your plot could be either a boxplot or a plot with one dot for
each hot dog.

    \begin{Verbatim}[commandchars=\\\{\}]
{\color{incolor}In [{\color{incolor}35}]:} hotdog\PY{o}{=}read.table\PY{p}{(}\PY{l+s}{\PYZdq{}}\PY{l+s}{hotdogs.txt\PYZdq{}}\PY{p}{,}header\PY{o}{=}\PY{k+kc}{TRUE}\PY{p}{)}
         ggplot\PY{p}{(}hotdog\PY{p}{,} aes\PY{p}{(}x\PY{o}{=}Type\PY{p}{,} y\PY{o}{=}Calories\PY{p}{,} color\PY{o}{=}Type\PY{p}{)}\PY{p}{)} \PY{o}{+}
         geom\PYZus{}boxplot\PY{p}{(}\PY{p}{)} \PY{o}{+}
         ylab\PY{p}{(}\PY{l+s}{\PYZsq{}}\PY{l+s}{calories\PYZsq{}}\PY{p}{)} \PY{o}{+}
         ggtitle\PY{p}{(}\PY{l+s}{\PYZsq{}}\PY{l+s}{Boxplots of calories of different protein types\PYZsq{}}\PY{p}{)} \PY{o}{+}
         theme\PY{p}{(}plot.title \PY{o}{=} element\PYZus{}text\PY{p}{(}hjust \PY{o}{=} \PY{l+m}{0.5}\PY{p}{)}\PY{p}{)} \PY{o}{+}
         ylim\PY{p}{(}\PY{k+kp}{min}\PY{p}{(}hotdog\PY{o}{\PYZdl{}}Calories\PY{p}{)}\PY{l+m}{\PYZhy{}0.05}\PY{p}{,} \PY{k+kp}{max}\PY{p}{(}hotdog\PY{o}{\PYZdl{}}Calories\PY{p}{)}\PY{l+m}{+0.05}\PY{p}{)}
\end{Verbatim}

    
    
    \begin{center}
    \adjustimage{max size={0.7\linewidth}{0.7\paperheight}}{output_48_1.png}
    \end{center}
    { \hspace*{\fill} \\}
    
\noindent    Answer the following question: ``Are there differences in the average
calories of hot dogs made with different kinds of meat?''. To answer
this question, write down a statistical model (clearly state the
response variable, treatment levels, number of replicates, . . . ),
express the above question as a testable null hypothesis, and report the
p-value of the test statistic under the null hypothesis. Conduct an
analysis of pairwise differences if it helps you clarify where there are
differences in mean calories. Your answer should include all R code
used, and the important R output. If you need to transform the response
variable, do so, but you do NOT need to provide details of the
transformations that you tried, but did not ultimately select. Only
report the best transformation, and only show residual plots, ANOVA
tables, and any other results for this transformation.

    \paragraph{Model and Null Hypothesis}\label{model-and-null-hypothesis}

    \[Y_{it}=\mu +\tau_i+\epsilon_{it},  \text{where } i=beef,pork,chicken\]

\[\epsilon \stackrel{iid}{\sim} N(0, \sigma^2)\]

and the number of replicates:

\[t_{beef}= 20,t_{pork}= 17,t_{chicken}=17\]

we need to test:

\[H_0: \tau _i= 0 \text{ for } i=beef,pork,chicken\]

    \paragraph{Check Whether Transformation is
Needed}\label{check-whether-transformation-is-needed}

    \begin{Verbatim}[commandchars=\\\{\}]
{\color{incolor}In [{\color{incolor}46}]:} par\PY{p}{(}mfrow\PY{o}{=}\PY{k+kt}{c}\PY{p}{(}\PY{l+m}{2}\PY{p}{,}\PY{l+m}{2}\PY{p}{)}\PY{p}{)}
	aov.hotdog \PY{o}{=}aov\PY{p}{(}Calories\PY{o}{\PYZti{}}Type\PY{p}{,} data\PY{o}{=}hotdog\PY{p}{)}
         plot\PY{p}{(}aov.hotdog\PY{p}{)}
\end{Verbatim}

    \begin{center}
    \adjustimage{max size={0.7\linewidth}{0.7\paperheight}}{output_53_0.png}
    \end{center}
    { \hspace*{\fill} \\}
    
    \begin{itemize}
\item
  From the QQ-plot above we could conclude that basically all the points
  fall on the dotted line, thus the residuals are approximately normal
\item
  From the Residual vs. Fitted plot we can see that for each vertical
  line of points representing a different treatment, the spread on the
  points seems evenly distributed around the horizontal baseline,
  indicating that these 3 treatments have the same variance. So the
  assumption of constant variance is not violated.
\end{itemize}

\textbf{Therefore no transformation needed.}

    \paragraph{ANOVA}\label{anova}

    \begin{Verbatim}[commandchars=\\\{\}]
{\color{incolor}In [{\color{incolor}43}]:}kable\PY{p}{(}anova\PY{p}{(}aov.hotdog\PY{p}{)}\PY{p}{,} format\PY{o}{=}\PY{l+s}{\PYZsq{}}\PY{l+s}{markdown\PYZsq{}}\PY{p}{)}
        \PY{l+m}{3.9e\PYZhy{}06} \PY{o}{\PYZlt{}} \PY{l+m}{0.5}
\end{Verbatim}

    
    \begin{verbatim}


|          | Df|   Sum Sq|  Mean Sq|  F value|  Pr(>F)|
|:---------|--:|--------:|--------:|--------:|-------:|
|Type      |  2| 17692.20| 8846.098| 16.07399| 3.9e-06|
|Residuals | 51| 28067.14|  550.336|       NA|      NA|
    \end{verbatim}

    
    TRUE

    
    so with an F-test from ANOVA table, we can see that since p-value =
3.862e-06\( \textless{} 0.5\), hence with statistical significance at
the 0.05 level, we could conclude that there is difference in calories
of different hotdogs.

    \paragraph{Pairwise Difference}\label{pairwise-difference}

    \begin{Verbatim}[commandchars=\\\{\}]
{\color{incolor}In [{\color{incolor}47}]:} lsm.hotdog\PY{o}{=}lsmeans\PY{p}{(}aov.hotdog\PY{p}{,} \PY{o}{\PYZti{}}Type\PY{p}{)}
         \PY{k+kp}{summary}\PY{p}{(}contrast\PY{p}{(}lsm.hotdog\PY{p}{,} method\PY{o}{=}\PY{l+s}{\PYZdq{}}\PY{l+s}{pairwise\PYZdq{}}\PY{p}{,} 
                          adjust\PY{o}{=}\PY{l+s}{\PYZdq{}}\PY{l+s}{tukey\PYZdq{}}\PY{p}{)}\PY{p}{,}infer\PY{o}{=}\PY{k+kt}{c}\PY{p}{(}\PY{n+nb+bp}{T}\PY{p}{,}\PY{n+nb+bp}{T}\PY{p}{)}\PY{p}{,} 
                          level\PY{o}{=}\PY{l+m}{0.95}\PY{p}{,} side\PY{o}{=}\PY{l+s}{\PYZdq{}}\PY{l+s}{two\PYZhy{}sided\PYZdq{}}\PY{p}{)}
         
         Beef\PYZus{}Chicken\PYZus{}pv \PY{o}{=} \PY{l+m}{2.767694e\PYZhy{}05}
         Beef\PYZus{}Pork\PYZus{}pv \PY{o}{=} \PY{l+m}{9.688129e\PYZhy{}01}
         Chicken\PYZus{}Pork\PYZus{}pv \PY{o}{=} \PY{l+m}{2.390087e\PYZhy{}05}
         
         Beef\PYZus{}Chicken\PYZus{}pv \PY{o}{\PYZlt{}} \PY{l+m}{0.5}
         Beef\PYZus{}Pork\PYZus{}pv \PY{o}{\PYZlt{}} \PY{l+m}{0.5}
         Chicken\PYZus{}Pork\PYZus{}pv \PY{o}{\PYZlt{}} \PY{l+m}{0.5}
\end{Verbatim}

    \begin{tabular}{r|llllllll}
 contrast & estimate & SE & df & lower.CL & upper.CL & t.ratio & p.value\\
\hline
	 Beef - Chicken &  38.085294     & 7.738831       & 51             &  19.40391      &  56.76667      &  4.9213237     & 2.767694e-05  \\
	 Beef - Pork    &  -1.855882     & 7.738831       & 51             & -20.53726      &  16.82550      & -0.2398143     & 9.688129e-01  \\
	 Chicken - Pork & -39.941176     & 8.046454       & 51             & -59.36515      & -20.51720      & -4.9638236     & 2.390087e-05  \\
\end{tabular}


    
    TRUE

    
    FALSE

    
    TRUE

    
    We can interpret the results of these tests with the following
statements:

\begin{itemize}
\item
  \begin{enumerate}
  \def\labelenumi{\arabic{enumi}.}
  \tightlist
  \item
    The calories of beef is not significantly different from the
    calories of pork;
  \end{enumerate}
\item
  \begin{enumerate}
  \def\labelenumi{\arabic{enumi}.}
  \setcounter{enumi}{1}
  \tightlist
  \item
    The calories of beef is significantly more than the calories of
    chicken;
  \end{enumerate}
\item
  \begin{enumerate}
  \def\labelenumi{\arabic{enumi}.}
  \setcounter{enumi}{2}
  \tightlist
  \item
    The calories of chicken is significantly less than the calories of
    pork;
  \end{enumerate}
\end{itemize}


    % Add a bibliography block to the postdoc
    
    
    
    \end{document}
