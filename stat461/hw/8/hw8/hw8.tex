
% Default to the notebook output style

    


% Inherit from the specified cell style.




    
\documentclass[11pt]{article}

    
    
    \usepackage[T1]{fontenc}
    % Nicer default font (+ math font) than Computer Modern for most use cases
    \usepackage{mathpazo}

    % Basic figure setup, for now with no caption control since it's done
    % automatically by Pandoc (which extracts ![](path) syntax from Markdown).
    \usepackage{graphicx}
    % We will generate all images so they have a width \maxwidth. This means
    % that they will get their normal width if they fit onto the page, but
    % are scaled down if they would overflow the margins.
    \makeatletter
    \def\maxwidth{\ifdim\Gin@nat@width>\linewidth\linewidth
    \else\Gin@nat@width\fi}
    \makeatother
    \let\Oldincludegraphics\includegraphics
    % Set max figure width to be 80% of text width, for now hardcoded.
    \renewcommand{\includegraphics}[1]{\Oldincludegraphics[width=.8\maxwidth]{#1}}
    % Ensure that by default, figures have no caption (until we provide a
    % proper Figure object with a Caption API and a way to capture that
    % in the conversion process - todo).
    \usepackage{caption}
    \DeclareCaptionLabelFormat{nolabel}{}
    \captionsetup{labelformat=nolabel}

    \usepackage{adjustbox} % Used to constrain images to a maximum size 
    \usepackage{xcolor} % Allow colors to be defined
    \usepackage{enumerate} % Needed for markdown enumerations to work
    \usepackage{geometry} % Used to adjust the document margins
    \usepackage{amsmath} % Equations
    \usepackage{amssymb} % Equations
    \usepackage{textcomp} % defines textquotesingle
    % Hack from http://tex.stackexchange.com/a/47451/13684:
    \AtBeginDocument{%
        \def\PYZsq{\textquotesingle}% Upright quotes in Pygmentized code
    }
    \usepackage{upquote} % Upright quotes for verbatim code
    \usepackage{eurosym} % defines \euro
    \usepackage[mathletters]{ucs} % Extended unicode (utf-8) support
    \usepackage[utf8x]{inputenc} % Allow utf-8 characters in the tex document
    \usepackage{fancyvrb} % verbatim replacement that allows latex
    \usepackage{grffile} % extends the file name processing of package graphics 
                         % to support a larger range 
    % The hyperref package gives us a pdf with properly built
    % internal navigation ('pdf bookmarks' for the table of contents,
    % internal cross-reference links, web links for URLs, etc.)
    \usepackage{hyperref}
    \usepackage{longtable} % longtable support required by pandoc >1.10
    \usepackage{booktabs}  % table support for pandoc > 1.12.2
    \usepackage[inline]{enumitem} % IRkernel/repr support (it uses the enumerate* environment)
    \usepackage[normalem]{ulem} % ulem is needed to support strikethroughs (\sout)
                                % normalem makes italics be italics, not underlines
    \usepackage{mathrsfs}
    

    
    
    % Colors for the hyperref package
    \definecolor{urlcolor}{rgb}{0,.145,.698}
    \definecolor{linkcolor}{rgb}{.71,0.21,0.01}
    \definecolor{citecolor}{rgb}{.12,.54,.11}

    % ANSI colors
    \definecolor{ansi-black}{HTML}{3E424D}
    \definecolor{ansi-black-intense}{HTML}{282C36}
    \definecolor{ansi-red}{HTML}{E75C58}
    \definecolor{ansi-red-intense}{HTML}{B22B31}
    \definecolor{ansi-green}{HTML}{00A250}
    \definecolor{ansi-green-intense}{HTML}{007427}
    \definecolor{ansi-yellow}{HTML}{DDB62B}
    \definecolor{ansi-yellow-intense}{HTML}{B27D12}
    \definecolor{ansi-blue}{HTML}{208FFB}
    \definecolor{ansi-blue-intense}{HTML}{0065CA}
    \definecolor{ansi-magenta}{HTML}{D160C4}
    \definecolor{ansi-magenta-intense}{HTML}{A03196}
    \definecolor{ansi-cyan}{HTML}{60C6C8}
    \definecolor{ansi-cyan-intense}{HTML}{258F8F}
    \definecolor{ansi-white}{HTML}{C5C1B4}
    \definecolor{ansi-white-intense}{HTML}{A1A6B2}
    \definecolor{ansi-default-inverse-fg}{HTML}{FFFFFF}
    \definecolor{ansi-default-inverse-bg}{HTML}{000000}

    % commands and environments needed by pandoc snippets
    % extracted from the output of `pandoc -s`
    \providecommand{\tightlist}{%
      \setlength{\itemsep}{0pt}\setlength{\parskip}{0pt}}
    \DefineVerbatimEnvironment{Highlighting}{Verbatim}{commandchars=\\\{\}}
    % Add ',fontsize=\small' for more characters per line
    \newenvironment{Shaded}{}{}
    \newcommand{\KeywordTok}[1]{\textcolor[rgb]{0.00,0.44,0.13}{\textbf{{#1}}}}
    \newcommand{\DataTypeTok}[1]{\textcolor[rgb]{0.56,0.13,0.00}{{#1}}}
    \newcommand{\DecValTok}[1]{\textcolor[rgb]{0.25,0.63,0.44}{{#1}}}
    \newcommand{\BaseNTok}[1]{\textcolor[rgb]{0.25,0.63,0.44}{{#1}}}
    \newcommand{\FloatTok}[1]{\textcolor[rgb]{0.25,0.63,0.44}{{#1}}}
    \newcommand{\CharTok}[1]{\textcolor[rgb]{0.25,0.44,0.63}{{#1}}}
    \newcommand{\StringTok}[1]{\textcolor[rgb]{0.25,0.44,0.63}{{#1}}}
    \newcommand{\CommentTok}[1]{\textcolor[rgb]{0.38,0.63,0.69}{\textit{{#1}}}}
    \newcommand{\OtherTok}[1]{\textcolor[rgb]{0.00,0.44,0.13}{{#1}}}
    \newcommand{\AlertTok}[1]{\textcolor[rgb]{1.00,0.00,0.00}{\textbf{{#1}}}}
    \newcommand{\FunctionTok}[1]{\textcolor[rgb]{0.02,0.16,0.49}{{#1}}}
    \newcommand{\RegionMarkerTok}[1]{{#1}}
    \newcommand{\ErrorTok}[1]{\textcolor[rgb]{1.00,0.00,0.00}{\textbf{{#1}}}}
    \newcommand{\NormalTok}[1]{{#1}}
    
    % Additional commands for more recent versions of Pandoc
    \newcommand{\ConstantTok}[1]{\textcolor[rgb]{0.53,0.00,0.00}{{#1}}}
    \newcommand{\SpecialCharTok}[1]{\textcolor[rgb]{0.25,0.44,0.63}{{#1}}}
    \newcommand{\VerbatimStringTok}[1]{\textcolor[rgb]{0.25,0.44,0.63}{{#1}}}
    \newcommand{\SpecialStringTok}[1]{\textcolor[rgb]{0.73,0.40,0.53}{{#1}}}
    \newcommand{\ImportTok}[1]{{#1}}
    \newcommand{\DocumentationTok}[1]{\textcolor[rgb]{0.73,0.13,0.13}{\textit{{#1}}}}
    \newcommand{\AnnotationTok}[1]{\textcolor[rgb]{0.38,0.63,0.69}{\textbf{\textit{{#1}}}}}
    \newcommand{\CommentVarTok}[1]{\textcolor[rgb]{0.38,0.63,0.69}{\textbf{\textit{{#1}}}}}
    \newcommand{\VariableTok}[1]{\textcolor[rgb]{0.10,0.09,0.49}{{#1}}}
    \newcommand{\ControlFlowTok}[1]{\textcolor[rgb]{0.00,0.44,0.13}{\textbf{{#1}}}}
    \newcommand{\OperatorTok}[1]{\textcolor[rgb]{0.40,0.40,0.40}{{#1}}}
    \newcommand{\BuiltInTok}[1]{{#1}}
    \newcommand{\ExtensionTok}[1]{{#1}}
    \newcommand{\PreprocessorTok}[1]{\textcolor[rgb]{0.74,0.48,0.00}{{#1}}}
    \newcommand{\AttributeTok}[1]{\textcolor[rgb]{0.49,0.56,0.16}{{#1}}}
    \newcommand{\InformationTok}[1]{\textcolor[rgb]{0.38,0.63,0.69}{\textbf{\textit{{#1}}}}}
    \newcommand{\WarningTok}[1]{\textcolor[rgb]{0.38,0.63,0.69}{\textbf{\textit{{#1}}}}}
    
    
    % Define a nice break command that doesn't care if a line doesn't already
    % exist.
    \def\br{\hspace*{\fill} \\* }
    % Math Jax compatibility definitions
    \def\gt{>}
    \def\lt{<}
    \let\Oldtex\TeX
    \let\Oldlatex\LaTeX
    \renewcommand{\TeX}{\textrm{\Oldtex}}
    \renewcommand{\LaTeX}{\textrm{\Oldlatex}}
    % Document parameters
    % Document title
    \title{Assignment 8}
    
    \author{Jiarong Ye}
    
    
    
    
    

    % Pygments definitions
    
\makeatletter
\def\PY@reset{\let\PY@it=\relax \let\PY@bf=\relax%
    \let\PY@ul=\relax \let\PY@tc=\relax%
    \let\PY@bc=\relax \let\PY@ff=\relax}
\def\PY@tok#1{\csname PY@tok@#1\endcsname}
\def\PY@toks#1+{\ifx\relax#1\empty\else%
    \PY@tok{#1}\expandafter\PY@toks\fi}
\def\PY@do#1{\PY@bc{\PY@tc{\PY@ul{%
    \PY@it{\PY@bf{\PY@ff{#1}}}}}}}
\def\PY#1#2{\PY@reset\PY@toks#1+\relax+\PY@do{#2}}

\expandafter\def\csname PY@tok@w\endcsname{\def\PY@tc##1{\textcolor[rgb]{0.73,0.73,0.73}{##1}}}
\expandafter\def\csname PY@tok@c\endcsname{\let\PY@it=\textit\def\PY@tc##1{\textcolor[rgb]{0.25,0.50,0.50}{##1}}}
\expandafter\def\csname PY@tok@cp\endcsname{\def\PY@tc##1{\textcolor[rgb]{0.74,0.48,0.00}{##1}}}
\expandafter\def\csname PY@tok@k\endcsname{\let\PY@bf=\textbf\def\PY@tc##1{\textcolor[rgb]{0.00,0.50,0.00}{##1}}}
\expandafter\def\csname PY@tok@kp\endcsname{\def\PY@tc##1{\textcolor[rgb]{0.00,0.50,0.00}{##1}}}
\expandafter\def\csname PY@tok@kt\endcsname{\def\PY@tc##1{\textcolor[rgb]{0.69,0.00,0.25}{##1}}}
\expandafter\def\csname PY@tok@o\endcsname{\def\PY@tc##1{\textcolor[rgb]{0.40,0.40,0.40}{##1}}}
\expandafter\def\csname PY@tok@ow\endcsname{\let\PY@bf=\textbf\def\PY@tc##1{\textcolor[rgb]{0.67,0.13,1.00}{##1}}}
\expandafter\def\csname PY@tok@nb\endcsname{\def\PY@tc##1{\textcolor[rgb]{0.00,0.50,0.00}{##1}}}
\expandafter\def\csname PY@tok@nf\endcsname{\def\PY@tc##1{\textcolor[rgb]{0.00,0.00,1.00}{##1}}}
\expandafter\def\csname PY@tok@nc\endcsname{\let\PY@bf=\textbf\def\PY@tc##1{\textcolor[rgb]{0.00,0.00,1.00}{##1}}}
\expandafter\def\csname PY@tok@nn\endcsname{\let\PY@bf=\textbf\def\PY@tc##1{\textcolor[rgb]{0.00,0.00,1.00}{##1}}}
\expandafter\def\csname PY@tok@ne\endcsname{\let\PY@bf=\textbf\def\PY@tc##1{\textcolor[rgb]{0.82,0.25,0.23}{##1}}}
\expandafter\def\csname PY@tok@nv\endcsname{\def\PY@tc##1{\textcolor[rgb]{0.10,0.09,0.49}{##1}}}
\expandafter\def\csname PY@tok@no\endcsname{\def\PY@tc##1{\textcolor[rgb]{0.53,0.00,0.00}{##1}}}
\expandafter\def\csname PY@tok@nl\endcsname{\def\PY@tc##1{\textcolor[rgb]{0.63,0.63,0.00}{##1}}}
\expandafter\def\csname PY@tok@ni\endcsname{\let\PY@bf=\textbf\def\PY@tc##1{\textcolor[rgb]{0.60,0.60,0.60}{##1}}}
\expandafter\def\csname PY@tok@na\endcsname{\def\PY@tc##1{\textcolor[rgb]{0.49,0.56,0.16}{##1}}}
\expandafter\def\csname PY@tok@nt\endcsname{\let\PY@bf=\textbf\def\PY@tc##1{\textcolor[rgb]{0.00,0.50,0.00}{##1}}}
\expandafter\def\csname PY@tok@nd\endcsname{\def\PY@tc##1{\textcolor[rgb]{0.67,0.13,1.00}{##1}}}
\expandafter\def\csname PY@tok@s\endcsname{\def\PY@tc##1{\textcolor[rgb]{0.73,0.13,0.13}{##1}}}
\expandafter\def\csname PY@tok@sd\endcsname{\let\PY@it=\textit\def\PY@tc##1{\textcolor[rgb]{0.73,0.13,0.13}{##1}}}
\expandafter\def\csname PY@tok@si\endcsname{\let\PY@bf=\textbf\def\PY@tc##1{\textcolor[rgb]{0.73,0.40,0.53}{##1}}}
\expandafter\def\csname PY@tok@se\endcsname{\let\PY@bf=\textbf\def\PY@tc##1{\textcolor[rgb]{0.73,0.40,0.13}{##1}}}
\expandafter\def\csname PY@tok@sr\endcsname{\def\PY@tc##1{\textcolor[rgb]{0.73,0.40,0.53}{##1}}}
\expandafter\def\csname PY@tok@ss\endcsname{\def\PY@tc##1{\textcolor[rgb]{0.10,0.09,0.49}{##1}}}
\expandafter\def\csname PY@tok@sx\endcsname{\def\PY@tc##1{\textcolor[rgb]{0.00,0.50,0.00}{##1}}}
\expandafter\def\csname PY@tok@m\endcsname{\def\PY@tc##1{\textcolor[rgb]{0.40,0.40,0.40}{##1}}}
\expandafter\def\csname PY@tok@gh\endcsname{\let\PY@bf=\textbf\def\PY@tc##1{\textcolor[rgb]{0.00,0.00,0.50}{##1}}}
\expandafter\def\csname PY@tok@gu\endcsname{\let\PY@bf=\textbf\def\PY@tc##1{\textcolor[rgb]{0.50,0.00,0.50}{##1}}}
\expandafter\def\csname PY@tok@gd\endcsname{\def\PY@tc##1{\textcolor[rgb]{0.63,0.00,0.00}{##1}}}
\expandafter\def\csname PY@tok@gi\endcsname{\def\PY@tc##1{\textcolor[rgb]{0.00,0.63,0.00}{##1}}}
\expandafter\def\csname PY@tok@gr\endcsname{\def\PY@tc##1{\textcolor[rgb]{1.00,0.00,0.00}{##1}}}
\expandafter\def\csname PY@tok@ge\endcsname{\let\PY@it=\textit}
\expandafter\def\csname PY@tok@gs\endcsname{\let\PY@bf=\textbf}
\expandafter\def\csname PY@tok@gp\endcsname{\let\PY@bf=\textbf\def\PY@tc##1{\textcolor[rgb]{0.00,0.00,0.50}{##1}}}
\expandafter\def\csname PY@tok@go\endcsname{\def\PY@tc##1{\textcolor[rgb]{0.53,0.53,0.53}{##1}}}
\expandafter\def\csname PY@tok@gt\endcsname{\def\PY@tc##1{\textcolor[rgb]{0.00,0.27,0.87}{##1}}}
\expandafter\def\csname PY@tok@err\endcsname{\def\PY@bc##1{\setlength{\fboxsep}{0pt}\fcolorbox[rgb]{1.00,0.00,0.00}{1,1,1}{\strut ##1}}}
\expandafter\def\csname PY@tok@kc\endcsname{\let\PY@bf=\textbf\def\PY@tc##1{\textcolor[rgb]{0.00,0.50,0.00}{##1}}}
\expandafter\def\csname PY@tok@kd\endcsname{\let\PY@bf=\textbf\def\PY@tc##1{\textcolor[rgb]{0.00,0.50,0.00}{##1}}}
\expandafter\def\csname PY@tok@kn\endcsname{\let\PY@bf=\textbf\def\PY@tc##1{\textcolor[rgb]{0.00,0.50,0.00}{##1}}}
\expandafter\def\csname PY@tok@kr\endcsname{\let\PY@bf=\textbf\def\PY@tc##1{\textcolor[rgb]{0.00,0.50,0.00}{##1}}}
\expandafter\def\csname PY@tok@bp\endcsname{\def\PY@tc##1{\textcolor[rgb]{0.00,0.50,0.00}{##1}}}
\expandafter\def\csname PY@tok@fm\endcsname{\def\PY@tc##1{\textcolor[rgb]{0.00,0.00,1.00}{##1}}}
\expandafter\def\csname PY@tok@vc\endcsname{\def\PY@tc##1{\textcolor[rgb]{0.10,0.09,0.49}{##1}}}
\expandafter\def\csname PY@tok@vg\endcsname{\def\PY@tc##1{\textcolor[rgb]{0.10,0.09,0.49}{##1}}}
\expandafter\def\csname PY@tok@vi\endcsname{\def\PY@tc##1{\textcolor[rgb]{0.10,0.09,0.49}{##1}}}
\expandafter\def\csname PY@tok@vm\endcsname{\def\PY@tc##1{\textcolor[rgb]{0.10,0.09,0.49}{##1}}}
\expandafter\def\csname PY@tok@sa\endcsname{\def\PY@tc##1{\textcolor[rgb]{0.73,0.13,0.13}{##1}}}
\expandafter\def\csname PY@tok@sb\endcsname{\def\PY@tc##1{\textcolor[rgb]{0.73,0.13,0.13}{##1}}}
\expandafter\def\csname PY@tok@sc\endcsname{\def\PY@tc##1{\textcolor[rgb]{0.73,0.13,0.13}{##1}}}
\expandafter\def\csname PY@tok@dl\endcsname{\def\PY@tc##1{\textcolor[rgb]{0.73,0.13,0.13}{##1}}}
\expandafter\def\csname PY@tok@s2\endcsname{\def\PY@tc##1{\textcolor[rgb]{0.73,0.13,0.13}{##1}}}
\expandafter\def\csname PY@tok@sh\endcsname{\def\PY@tc##1{\textcolor[rgb]{0.73,0.13,0.13}{##1}}}
\expandafter\def\csname PY@tok@s1\endcsname{\def\PY@tc##1{\textcolor[rgb]{0.73,0.13,0.13}{##1}}}
\expandafter\def\csname PY@tok@mb\endcsname{\def\PY@tc##1{\textcolor[rgb]{0.40,0.40,0.40}{##1}}}
\expandafter\def\csname PY@tok@mf\endcsname{\def\PY@tc##1{\textcolor[rgb]{0.40,0.40,0.40}{##1}}}
\expandafter\def\csname PY@tok@mh\endcsname{\def\PY@tc##1{\textcolor[rgb]{0.40,0.40,0.40}{##1}}}
\expandafter\def\csname PY@tok@mi\endcsname{\def\PY@tc##1{\textcolor[rgb]{0.40,0.40,0.40}{##1}}}
\expandafter\def\csname PY@tok@il\endcsname{\def\PY@tc##1{\textcolor[rgb]{0.40,0.40,0.40}{##1}}}
\expandafter\def\csname PY@tok@mo\endcsname{\def\PY@tc##1{\textcolor[rgb]{0.40,0.40,0.40}{##1}}}
\expandafter\def\csname PY@tok@ch\endcsname{\let\PY@it=\textit\def\PY@tc##1{\textcolor[rgb]{0.25,0.50,0.50}{##1}}}
\expandafter\def\csname PY@tok@cm\endcsname{\let\PY@it=\textit\def\PY@tc##1{\textcolor[rgb]{0.25,0.50,0.50}{##1}}}
\expandafter\def\csname PY@tok@cpf\endcsname{\let\PY@it=\textit\def\PY@tc##1{\textcolor[rgb]{0.25,0.50,0.50}{##1}}}
\expandafter\def\csname PY@tok@c1\endcsname{\let\PY@it=\textit\def\PY@tc##1{\textcolor[rgb]{0.25,0.50,0.50}{##1}}}
\expandafter\def\csname PY@tok@cs\endcsname{\let\PY@it=\textit\def\PY@tc##1{\textcolor[rgb]{0.25,0.50,0.50}{##1}}}

\def\PYZbs{\char`\\}
\def\PYZus{\char`\_}
\def\PYZob{\char`\{}
\def\PYZcb{\char`\}}
\def\PYZca{\char`\^}
\def\PYZam{\char`\&}
\def\PYZlt{\char`\<}
\def\PYZgt{\char`\>}
\def\PYZsh{\char`\#}
\def\PYZpc{\char`\%}
\def\PYZdl{\char`\$}
\def\PYZhy{\char`\-}
\def\PYZsq{\char`\'}
\def\PYZdq{\char`\"}
\def\PYZti{\char`\~}
% for compatibility with earlier versions
\def\PYZat{@}
\def\PYZlb{[}
\def\PYZrb{]}
\makeatother


    % Exact colors from NB
    \definecolor{incolor}{rgb}{0.0, 0.0, 0.5}
    \definecolor{outcolor}{rgb}{0.545, 0.0, 0.0}



    
    % Prevent overflowing lines due to hard-to-break entities
    \sloppy 
    % Setup hyperref package
    \hypersetup{
      breaklinks=true,  % so long urls are correctly broken across lines
      colorlinks=true,
      urlcolor=urlcolor,
      linkcolor=linkcolor,
      citecolor=citecolor,
      }
    % Slightly bigger margins than the latex defaults
    
    \geometry{verbose,tmargin=1in,bmargin=1in,lmargin=1in,rmargin=1in}
    
    

    \begin{document}
    
    
    \maketitle
    
    

    
    \subsection*{Q1}\label{q1}

\begin{enumerate}
\def\labelenumi{\arabic{enumi}.}
\tightlist
\item
  Greenhouse. Consider an experiment to study the effect of three types
  of fertilizer (F1, F2, and F3) on the growth of two species of plant
  (SppA and SppB). The data are as follows:
\end{enumerate}

    \subsubsection*{input data}\label{input-data}

    \begin{Verbatim}[commandchars=\\\{\}]
{\color{incolor}In [{\color{incolor}100}]:} Fert\PY{o}{=}\PY{k+kt}{c}\PY{p}{(}\PY{k+kp}{rep}\PY{p}{(}\PY{l+s}{\PYZdq{}}\PY{l+s}{control\PYZdq{}}\PY{p}{,} \PY{l+m}{12}\PY{p}{)}\PY{p}{,} \PY{k+kp}{rep}\PY{p}{(}\PY{l+s}{\PYZdq{}}\PY{l+s}{f1\PYZdq{}}\PY{p}{,} \PY{l+m}{12}\PY{p}{)}\PY{p}{,}
          \PY{k+kp}{rep}\PY{p}{(}\PY{l+s}{\PYZdq{}}\PY{l+s}{f2\PYZdq{}}\PY{p}{,} \PY{l+m}{12}\PY{p}{)}\PY{p}{,} \PY{k+kp}{rep}\PY{p}{(}\PY{l+s}{\PYZdq{}}\PY{l+s}{f3\PYZdq{}}\PY{p}{,} \PY{l+m}{12}\PY{p}{)}\PY{p}{)}
          Species\PY{o}{=}\PY{k+kt}{c}\PY{p}{(}\PY{k+kp}{rep}\PY{p}{(}\PY{k+kt}{c}\PY{p}{(}\PY{k+kp}{rep}\PY{p}{(}\PY{l+s}{\PYZdq{}}\PY{l+s}{SppA\PYZdq{}}\PY{p}{,} \PY{l+m}{6}\PY{p}{)}\PY{p}{,} \PY{k+kp}{rep}\PY{p}{(}\PY{l+s}{\PYZdq{}}\PY{l+s}{SppB\PYZdq{}}\PY{p}{,} \PY{l+m}{6}\PY{p}{)}\PY{p}{)}\PY{p}{,}\PY{l+m}{4}\PY{p}{)}\PY{p}{)}
          Height\PY{o}{=}\PY{k+kt}{c}\PY{p}{(}\PY{l+m}{21.0}\PY{p}{,} \PY{l+m}{19.5}\PY{p}{,} \PY{l+m}{22.5}\PY{p}{,} \PY{l+m}{21.5}\PY{p}{,} \PY{l+m}{20.5}\PY{p}{,} \PY{l+m}{21.0}\PY{p}{,}
          \PY{l+m}{23.7}\PY{p}{,} \PY{l+m}{23.8}\PY{p}{,} \PY{l+m}{23.8}\PY{p}{,} \PY{l+m}{23.7}\PY{p}{,} \PY{l+m}{22.8}\PY{p}{,} \PY{l+m}{24.4}\PY{p}{,}
          \PY{l+m}{32.0}\PY{p}{,} \PY{l+m}{30.5}\PY{p}{,} \PY{l+m}{25.0}\PY{p}{,} \PY{l+m}{27.5}\PY{p}{,} \PY{l+m}{28.0}\PY{p}{,} \PY{l+m}{28.6}\PY{p}{,}
          \PY{l+m}{30.1}\PY{p}{,} \PY{l+m}{28.9}\PY{p}{,} \PY{l+m}{30.9}\PY{p}{,} \PY{l+m}{34.4}\PY{p}{,} \PY{l+m}{32.7}\PY{p}{,} \PY{l+m}{32.7}\PY{p}{,}
          \PY{l+m}{22.5}\PY{p}{,} \PY{l+m}{26.0}\PY{p}{,} \PY{l+m}{28.0}\PY{p}{,} \PY{l+m}{27.0}\PY{p}{,} \PY{l+m}{26.5}\PY{p}{,} \PY{l+m}{25.2}\PY{p}{,}
          \PY{l+m}{30.6}\PY{p}{,} \PY{l+m}{31.1}\PY{p}{,} \PY{l+m}{28.1}\PY{p}{,} \PY{l+m}{34.9}\PY{p}{,} \PY{l+m}{30.1}\PY{p}{,} \PY{l+m}{25.5}\PY{p}{,}
          \PY{l+m}{28.0}\PY{p}{,} \PY{l+m}{27.5}\PY{p}{,} \PY{l+m}{31.0}\PY{p}{,} \PY{l+m}{29.5}\PY{p}{,} \PY{l+m}{30.0}\PY{p}{,} \PY{l+m}{29.2}\PY{p}{,}
          \PY{l+m}{36.1}\PY{p}{,} \PY{l+m}{36.6}\PY{p}{,} \PY{l+m}{38.7}\PY{p}{,} \PY{l+m}{37.1}\PY{p}{,} \PY{l+m}{36.8}\PY{p}{,} \PY{l+m}{37.1}\PY{p}{)}
          df\PY{o}{=}\PY{k+kt}{data.frame}\PY{p}{(}Fert\PY{o}{=}Fert\PY{p}{,} Species\PY{o}{=}Species\PY{p}{,} Height\PY{o}{=}Height\PY{p}{)}
          df
\end{Verbatim}

    \begin{tabular}{|r|l|l|l|}
 Fert & Species & Height\\
\hline
	 control & SppA    & 21.0   \\
	 control & SppA    & 19.5   \\
	 control & SppA    & 22.5   \\
	 control & SppA    & 21.5   \\
	 control & SppA    & 20.5   \\
	 control & SppA    & 21.0   \\
	 control & SppB    & 23.7   \\
	 control & SppB    & 23.8   \\
	 control & SppB    & 23.8   \\
	 control & SppB    & 23.7   \\
	 control & SppB    & 22.8   \\
	 control & SppB    & 24.4   \\
	 f1      & SppA    & 32.0   \\
	 f1      & SppA    & 30.5   \\
	 f1      & SppA    & 25.0   \\
	 f1      & SppA    & 27.5   \\
	 f1      & SppA    & 28.0   \\
	 f1      & SppA    & 28.6   \\
	 f1      & SppB    & 30.1   \\
	 f1      & SppB    & 28.9   \\
	 f1      & SppB    & 30.9   \\
	 f1      & SppB    & 34.4   \\
	 f1      & SppB    & 32.7   \\
	 f1      & SppB    & 32.7   \\
	 f2      & SppA    & 22.5   \\
	 f2      & SppA    & 26.0   \\
	 f2      & SppA    & 28.0   \\
	 f2      & SppA    & 27.0   \\
	 f2      & SppA    & 26.5   \\
	 f2      & SppA    & 25.2   \\
	 f2      & SppB    & 30.6   \\
	 f2      & SppB    & 31.1   \\
	 f2      & SppB    & 28.1   \\
	 f2      & SppB    & 34.9   \\
	 f2      & SppB    & 30.1   \\
	 f2      & SppB    & 25.5   \\
	 f3      & SppA    & 28.0   \\
	 f3      & SppA    & 27.5   \\
	 f3      & SppA    & 31.0   \\
	 f3      & SppA    & 29.5   \\
	 f3      & SppA    & 30.0   \\
	 f3      & SppA    & 29.2   \\
	 f3      & SppB    & 36.1   \\
	 f3      & SppB    & 36.6   \\
	 f3      & SppB    & 38.7   \\
	 f3      & SppB    & 37.1   \\
	 f3      & SppB    & 36.8   \\
	 f3      & SppB    & 37.1   \\
\end{tabular}


    
    \subsubsection*{(a)}\label{a}

Write out the 2-way complete model for this experiment.

\[Y_{ijt} = \mu + \alpha_i + \beta_j + (\alpha \beta)_{ij} + \epsilon_{ijt}, \: \epsilon_{ijt}
\stackrel{iid}{\sim} N(0, \sigma^2)\]

\[i = control, f_1, f_2, f_3\] \[j = SppA, SppB\]
\[t = 1, 2, 3, 4, 5, 6\]

    \subsubsection*{(b)}\label{b}

Fit the model using R and examine the residuals. Transform the response
if needed to address any problems with normality or constant error
variance. If you transform the response, clearly show the residuals from
the un-transformed response, and your best transformation, and describe
why you chose the transformation you did.

    \paragraph{Null hypothesis:}\label{null-hypothesis}

\[H_0 : (\alpha \beta)_{ij} = 0 \: for \: all \: i, j\]

    \begin{Verbatim}[commandchars=\\\{\}]
{\color{incolor}In [{\color{incolor}101}]:} \PY{k+kn}{library}\PY{p}{(}car\PY{p}{)}
          modelAB\PY{o}{=}aov\PY{p}{(}Height\PY{o}{\PYZti{}}Fert\PY{o}{+}Species\PY{o}{+}Fert\PY{o}{:}Species\PY{p}{,} data\PY{o}{=}df\PY{p}{)}
          anova\PY{p}{(}modelAB\PY{p}{)}
          Anova\PY{p}{(}modelAB\PY{p}{,}type\PY{o}{=}\PY{l+s}{\PYZdq{}}\PY{l+s}{III\PYZdq{}}\PY{p}{)}
\end{Verbatim}

\[\]

    \begin{tabular}{|r|l|l|l|l|l|}
  & Df & Sum Sq & Mean Sq & F value & Pr(>F)\\
\hline
	Fert &  3           & 745.43750    & 248.47917    & 73.098232    & 2.765659e-16\\
	Species &  1           & 236.74083    & 236.74083    & 69.645020    & 2.706508e-10\\
	Fert:Species &  3           &  50.58417    &  16.86139    &  4.960326    & 5.080577e-03\\
	Residuals & 40           & 135.97000    &   3.39925    &        NA    &           NA\\
\end{tabular}

\[\]
\[\]

    
    \begin{tabular}{|r|l|l|l|l|}
  & Sum Sq & Df & F value & Pr(>F)\\
\hline
	(Intercept) & 2646.00000   &  1           & 778.407002   & 7.768919e-28\\
	Fert &  251.44000   &  3           &  24.656419   & 3.373242e-09\\
	Species &   21.87000   &  1           &   6.433772   & 1.520353e-02\\
	Fert:Species &   50.58417   &  3           &   4.960326   & 5.080577e-03\\
	Residuals &  135.97000   & 40           &         NA   &           NA\\
\end{tabular}

\[\]
    
    \begin{Verbatim}[commandchars=\\\{\}]
{\color{incolor}In [{\color{incolor}98}]:} p\PYZus{}fert\PYZus{}species \PY{o}{=} \PY{l+m}{5.080577e\PYZhy{}03}
         p\PYZus{}fert\PYZus{}species \PY{o}{\PYZlt{}} \PY{l+m}{0.05}
\end{Verbatim}

    TRUE

\[\]
    
    Since the p value of \texttt{Fert:Species} are smaller than 0.05, hence
we are able to reject the null hypothesis at significance level of
\(\alpha = 0.05\), assume that interaction effects are not zero.

    \subsubsection*{check whether transformation is
needed}\label{check-whether-transformation-is-needed}

    \begin{Verbatim}[commandchars=\\\{\}]
{\color{incolor}In [{\color{incolor}102}]:} par\PY{p}{(}mfrow\PY{o}{=}\PY{k+kt}{c}\PY{p}{(}\PY{l+m}{2}\PY{p}{,}\PY{l+m}{2}\PY{p}{)}\PY{p}{)}
          plot\PY{p}{(}modelAB\PY{p}{)}
\end{Verbatim}

    \begin{center}
    \adjustimage{max size={0.9\linewidth}{0.9\paperheight}}{output_10_0.png}
    \end{center}
    { \hspace*{\fill} \\}
    
    From the Residual vs. Fitted plot we can see that for each vertical line
of points representing a different treatment, the spread on the points
does not appear to be equal. These 3 treatments do not have the same
variance, the 2nd treatment has slightly larger spread than the others.
So the assumption of constant variance is violated.

From the QQ-plot above we could conclude that since not all the points
fall on the dotted line, thus the residuals are not normal, it also
appears to be slightly heavy tailed.

    \subsubsection*{try transformation}\label{try-transformation}

    \begin{Verbatim}[commandchars=\\\{\}]
{\color{incolor}In [{\color{incolor}23}]:} \PY{k+kn}{library}\PY{p}{(}knitr\PY{p}{)}
         \PY{k+kn}{library}\PY{p}{(}lsmeans\PY{p}{)}
\end{Verbatim}

    \paragraph{square root}\label{square-root}

    \begin{Verbatim}[commandchars=\\\{\}]
{\color{incolor}In [{\color{incolor}103}]:} df\PY{o}{\PYZdl{}}sqrt\PYZus{}Height \PY{o}{=} \PY{k+kp}{sqrt}\PY{p}{(}df\PY{o}{\PYZdl{}}Height\PY{p}{)}
          model\PYZus{}sqrt\PYZus{}AB\PY{o}{=}aov\PY{p}{(}sqrt\PYZus{}Height\PY{o}{\PYZti{}}Fert\PY{o}{+}Species\PY{o}{+}Fert\PY{o}{:}Species\PY{p}{,} data\PY{o}{=}df\PY{p}{)}
          kable\PY{p}{(}anova\PY{p}{(}model\PYZus{}sqrt\PYZus{}AB\PY{p}{)}\PY{p}{,} format\PY{o}{=}\PY{l+s}{\PYZsq{}}\PY{l+s}{markdown\PYZsq{}}\PY{p}{)}
          par\PY{p}{(}mfrow\PY{o}{=}\PY{k+kt}{c}\PY{p}{(}\PY{l+m}{2}\PY{p}{,}\PY{l+m}{2}\PY{p}{)}\PY{p}{)}
          plot\PY{p}{(}model\PYZus{}sqrt\PYZus{}AB\PY{p}{)}
\end{Verbatim}

    
    \begin{verbatim}


|             | Df|    Sum Sq|   Mean Sq|   F value|    Pr(>F)|
|:------------|--:|---------:|---------:|---------:|---------:|
|Fert         |  3| 6.7501422| 2.2500474| 75.993806| 0.0000000|
|Species      |  1| 2.0196976| 2.0196976| 68.213898| 0.0000000|
|Fert:Species |  3| 0.3273222| 0.1091074|  3.685028| 0.0196204|
|Residuals    | 40| 1.1843320| 0.0296083|        NA|        NA|
    \end{verbatim}

    
    \begin{center}
    \adjustimage{max size={0.9\linewidth}{0.9\paperheight}}{output_15_1.png}
    \end{center}
    { \hspace*{\fill} \\}
    
    \paragraph{log}\label{log}

    \begin{Verbatim}[commandchars=\\\{\}]
{\color{incolor}In [{\color{incolor}104}]:} df\PY{o}{\PYZdl{}}log\PYZus{}Height \PY{o}{=} \PY{k+kp}{log}\PY{p}{(}df\PY{o}{\PYZdl{}}Height\PY{p}{)}
          model\PYZus{}log\PYZus{}AB\PY{o}{=}aov\PY{p}{(}log\PYZus{}Height\PY{o}{\PYZti{}}Fert\PY{o}{+}Species\PY{o}{+}Fert\PY{o}{:}Species\PY{p}{,} data\PY{o}{=}df\PY{p}{)}
          kable\PY{p}{(}anova\PY{p}{(}model\PYZus{}log\PYZus{}AB\PY{p}{)}\PY{p}{,} format\PY{o}{=}\PY{l+s}{\PYZsq{}}\PY{l+s}{markdown\PYZsq{}}\PY{p}{)}
          par\PY{p}{(}mfrow\PY{o}{=}\PY{k+kt}{c}\PY{p}{(}\PY{l+m}{2}\PY{p}{,}\PY{l+m}{2}\PY{p}{)}\PY{p}{)}
          plot\PY{p}{(}model\PYZus{}log\PYZus{}AB\PY{p}{)}
\end{Verbatim}

    
    \begin{verbatim}


|             | Df|    Sum Sq|   Mean Sq|   F value|    Pr(>F)|
|:------------|--:|---------:|---------:|---------:|---------:|
|Fert         |  3| 0.9902483| 0.3300828| 78.813105| 0.0000000|
|Species      |  1| 0.2791987| 0.2791987| 66.663622| 0.0000000|
|Fert:Species |  3| 0.0331922| 0.0110641|  2.641739| 0.0624076|
|Residuals    | 40| 0.1675268| 0.0041882|        NA|        NA|
    \end{verbatim}

    
    \begin{center}
    \adjustimage{max size={0.9\linewidth}{0.9\paperheight}}{output_17_1.png}
    \end{center}
    { \hspace*{\fill} \\}
    
    \paragraph{square}\label{square}

    \begin{Verbatim}[commandchars=\\\{\}]
{\color{incolor}In [{\color{incolor}105}]:} df\PY{o}{\PYZdl{}}square\PYZus{}Height \PY{o}{=} df\PY{o}{\PYZdl{}}Height\PY{o}{\PYZca{}}\PY{l+m}{2}
          model\PYZus{}square\PYZus{}AB\PY{o}{=}aov\PY{p}{(}square\PYZus{}Height\PY{o}{\PYZti{}}Fert\PY{o}{+}Species\PY{o}{+}Fert\PY{o}{:}Species\PY{p}{,} data\PY{o}{=}df\PY{p}{)}
          kable\PY{p}{(}anova\PY{p}{(}model\PYZus{}square\PYZus{}AB\PY{p}{)}\PY{p}{,} format\PY{o}{=}\PY{l+s}{\PYZsq{}}\PY{l+s}{markdown\PYZsq{}}\PY{p}{)}
          par\PY{p}{(}mfrow\PY{o}{=}\PY{k+kt}{c}\PY{p}{(}\PY{l+m}{2}\PY{p}{,}\PY{l+m}{2}\PY{p}{)}\PY{p}{)}
          plot\PY{p}{(}model\PYZus{}square\PYZus{}AB\PY{p}{)}
\end{Verbatim}

    
    \begin{verbatim}


|             | Df|    Sum Sq|   Mean Sq|  F value|    Pr(>F)|
|:------------|--:|---------:|---------:|--------:|---------:|
|Fert         |  3| 2362176.0| 787392.01| 67.49736| 0.0000000|
|Species      |  1|  842503.9| 842503.91| 72.22170| 0.0000000|
|Fert:Species |  3|  282582.2|  94194.08|  8.07457| 0.0002528|
|Residuals    | 40|  466620.9|  11665.52|       NA|        NA|
    \end{verbatim}

    
    \begin{center}
    \adjustimage{max size={0.9\linewidth}{0.9\paperheight}}{output_19_1.png}
    \end{center}
    { \hspace*{\fill} \\}
    
    However, it seems that after transformation, the residual vs. fitted
value plot and the QQ plot still remains as similar to the original
version without incurring much changes. Hence we could leave the Height
parameter untransformed.

    \subsubsection*{(c)}\label{c}

Describe the effect of species and fertilizer on mean height. This
description should use the results of hypothesis tests and p-values as
described in class. Discuss any relevant interaction effects, main
effects and pairwise differences between treatment means. Provide a plot
that shows the means for all combinations of factor levels. Provide R
code and output that supports your results.

    Since we reject the null hypothesis and find a significant interaction
effect between Fert and Species, so instead of looking at the levels of
Factor Fert and Factor Species independently, we proceed directly to
looking for significant differences between all possible combinations of
treatments.

    \begin{Verbatim}[commandchars=\\\{\}]
{\color{incolor}In [{\color{incolor}66}]:} \PY{k+kn}{library}\PY{p}{(}multcompView\PY{p}{)}
         modelAB\PY{o}{=}aov\PY{p}{(}Height\PY{o}{\PYZti{}}Fert\PY{o}{+}Species\PY{o}{+}Fert\PY{o}{:}Species\PY{p}{,} data\PY{o}{=}df\PY{p}{)}
         lsminter\PY{o}{=}lsmeans\PY{p}{(}modelAB\PY{p}{,} \PY{o}{\PYZti{}} Fert\PY{o}{:}Species\PY{p}{)}
         contrast\PY{p}{(}lsminter\PY{p}{,}method\PY{o}{=}\PY{l+s}{\PYZdq{}}\PY{l+s}{pairwise\PYZdq{}}\PY{p}{)}
         contrast\PYZus{}inter \PY{o}{=} cld\PY{p}{(}lsminter\PY{p}{,} alpha\PY{o}{=}\PY{l+m}{0.05}\PY{p}{)}
         contrast\PYZus{}inter
         plot\PY{p}{(}contrast\PYZus{}inter\PY{p}{)}
\end{Verbatim}

\[\]

    
    \begin{verbatim}
 contrast                      estimate       SE df t.ratio p.value
 control,SppA - f1,SppA       -7.600000 1.064464 40  -7.140  <.0001
 control,SppA - f2,SppA       -4.866667 1.064464 40  -4.572  0.0011
 control,SppA - f3,SppA       -8.200000 1.064464 40  -7.703  <.0001
 control,SppA - control,SppB  -2.700000 1.064464 40  -2.536  0.2101
 control,SppA - f1,SppB      -10.616667 1.064464 40  -9.974  <.0001
 control,SppA - f2,SppB       -9.050000 1.064464 40  -8.502  <.0001
 control,SppA - f3,SppB      -16.066667 1.064464 40 -15.094  <.0001
 f1,SppA - f2,SppA             2.733333 1.064464 40   2.568  0.1979
 f1,SppA - f3,SppA            -0.600000 1.064464 40  -0.564  0.9991
 f1,SppA - control,SppB        4.900000 1.064464 40   4.603  0.0010
 f1,SppA - f1,SppB            -3.016667 1.064464 40  -2.834  0.1150
 f1,SppA - f2,SppB            -1.450000 1.064464 40  -1.362  0.8685
 f1,SppA - f3,SppB            -8.466667 1.064464 40  -7.954  <.0001
 f2,SppA - f3,SppA            -3.333333 1.064464 40  -3.131  0.0585
 f2,SppA - control,SppB        2.166667 1.064464 40   2.035  0.4722
 f2,SppA - f1,SppB            -5.750000 1.064464 40  -5.402  0.0001
 f2,SppA - f2,SppB            -4.183333 1.064464 40  -3.930  0.0072
 f2,SppA - f3,SppB           -11.200000 1.064464 40 -10.522  <.0001
 f3,SppA - control,SppB        5.500000 1.064464 40   5.167  0.0002
 f3,SppA - f1,SppB            -2.416667 1.064464 40  -2.270  0.3345
 f3,SppA - f2,SppB            -0.850000 1.064464 40  -0.799  0.9922
 f3,SppA - f3,SppB            -7.866667 1.064464 40  -7.390  <.0001
 control,SppB - f1,SppB       -7.916667 1.064464 40  -7.437  <.0001
 control,SppB - f2,SppB       -6.350000 1.064464 40  -5.965  <.0001
 control,SppB - f3,SppB      -13.366667 1.064464 40 -12.557  <.0001
 f1,SppB - f2,SppB             1.566667 1.064464 40   1.472  0.8174
 f1,SppB - f3,SppB            -5.450000 1.064464 40  -5.120  0.0002
 f2,SppB - f3,SppB            -7.016667 1.064464 40  -6.592  <.0001

P value adjustment: tukey method for comparing a family of 8 estimates 
    \end{verbatim}

\[\]
    
    \begin{tabular}{|r|l|l|l|l|l|l|l|l|}
  & Fert & Species & lsmean & SE & df & lower.CL & upper.CL & .group\\
\hline
	1 & control   & SppA      & 21.00000  & 0.7526896 & 40        & 19.47876  & 22.52124  &  1       \\
	5 & control   & SppB      & 23.70000  & 0.7526896 & 40        & 22.17876  & 25.22124  &  12      \\
	3 & f2        & SppA      & 25.86667  & 0.7526896 & 40        & 24.34542  & 27.38791  &   23     \\
	2 & f1        & SppA      & 28.60000  & 0.7526896 & 40        & 27.07876  & 30.12124  &    34    \\
	4 & f3        & SppA      & 29.20000  & 0.7526896 & 40        & 27.67876  & 30.72124  &    34    \\
	7 & f2        & SppB      & 30.05000  & 0.7526896 & 40        & 28.52876  & 31.57124  &     4    \\
	6 & f1        & SppB      & 31.61667  & 0.7526896 & 40        & 30.09542  & 33.13791  &     4    \\
	8 & f3        & SppB      & 37.06667  & 0.7526896 & 40        & 35.54542  & 38.58791  &      5   \\
\end{tabular}


    
    
    
    \begin{center}
    \adjustimage{max size={0.7\linewidth}{0.9\paperheight}}{output_23_3.png}
    \end{center}
    { \hspace*{\fill} \\}
    
    We could interpret the results of this analysis as follows: 
    \begin{itemize}
    	\item control-SppA has significantly lower mean than f1-SppA , f3-SppA,
    	f1-SppB, f2-SppB, and f3-SppB.
    	\item control-SppB has significantly lower
    	mean than f1-SppB, f2-SppB, and f3-SppB.
    	\item f2-SppA has significantly
    	lower mean than f1-SppB, f2-SppB, and f3-SppB.
    	\item f1-SppA has
    	significantly lower mean than f3-SppB.
    	\item f3-SppA has significantly lower
    	mean than f3-SppB.
    	\item  f2-SppB has significantly lower mean than f3-SppB.
    	\item  f1-SppB has significantly lower mean than f3-SppB.
    	\item  No other
    	comparisons are significantly different than zero.
    \end{itemize}
    
    \[\]
       

    \subsection*{Q2}\label{q2}

\begin{enumerate}
\def\labelenumi{\arabic{enumi}.}
\setcounter{enumi}{1}
\tightlist
\item
  Consider the following data, the result of a 2-factor factorial
  experiment with 5 replications for each combination of Factor A and
  Factor B. Treatment combinations were assigned at random to the 20
  experimental units.
\end{enumerate}

    \subsubsection*{input data}\label{input-data}

    \begin{Verbatim}[commandchars=\\\{\}]
{\color{incolor}In [{\color{incolor}89}]:} A\PY{o}{=}\PY{k+kp}{as.factor}\PY{p}{(}\PY{k+kt}{c}\PY{p}{(}\PY{k+kp}{rep}\PY{p}{(}\PY{l+m}{1}\PY{p}{,} \PY{l+m}{10}\PY{p}{)}\PY{p}{,} \PY{k+kp}{rep}\PY{p}{(}\PY{l+m}{2}\PY{p}{,} \PY{l+m}{10}\PY{p}{)}\PY{p}{)}\PY{p}{)}
         B\PY{o}{=}\PY{k+kp}{as.factor}\PY{p}{(}\PY{k+kp}{rep}\PY{p}{(}\PY{k+kt}{c}\PY{p}{(}\PY{k+kt}{c}\PY{p}{(}\PY{k+kp}{rep}\PY{p}{(}\PY{l+m}{1}\PY{p}{,} \PY{l+m}{5}\PY{p}{)}\PY{p}{,} \PY{k+kp}{rep}\PY{p}{(}\PY{l+m}{2}\PY{p}{,} \PY{l+m}{5}\PY{p}{)}\PY{p}{)}\PY{p}{)}\PY{p}{,} \PY{l+m}{2}\PY{p}{)}\PY{p}{)}
         resp\PY{o}{=}\PY{k+kt}{c}\PY{p}{(}\PY{l+m}{12.9}\PY{p}{,} \PY{l+m}{11.3}\PY{p}{,} \PY{l+m}{11.7}\PY{p}{,} \PY{l+m}{12.1}\PY{p}{,} \PY{l+m}{12.3}\PY{p}{,}
         \PY{l+m}{13.7}\PY{p}{,} \PY{l+m}{12.8}\PY{p}{,} \PY{l+m}{13.6}\PY{p}{,} \PY{l+m}{13.1}\PY{p}{,} \PY{l+m}{13.5}\PY{p}{,}
         \PY{l+m}{14.2}\PY{p}{,} \PY{l+m}{14.5}\PY{p}{,} \PY{l+m}{13.9}\PY{p}{,} \PY{l+m}{13.6}\PY{p}{,} \PY{l+m}{14.4}\PY{p}{,}
         \PY{l+m}{13.5}\PY{p}{,} \PY{l+m}{13.1}\PY{p}{,} \PY{l+m}{13.3}\PY{p}{,} \PY{l+m}{13.1}\PY{p}{,} \PY{l+m}{13.4}\PY{p}{)}
         df\PY{o}{=}\PY{k+kt}{data.frame}\PY{p}{(}A\PY{o}{=}A\PY{p}{,} B\PY{o}{=}B\PY{p}{,} resp\PY{o}{=}resp\PY{p}{)}
         df
\end{Verbatim}

    \begin{tabular}{|r|l|l|l|}
 A & B & resp\\
\hline
	 1    & 1    & 12.9\\
	 1    & 1    & 11.3\\
	 1    & 1    & 11.7\\
	 1    & 1    & 12.1\\
	 1    & 1    & 12.3\\
	 1    & 2    & 13.7\\
	 1    & 2    & 12.8\\
	 1    & 2    & 13.6\\
	 1    & 2    & 13.1\\
	 1    & 2    & 13.5\\
	 2    & 1    & 14.2\\
	 2    & 1    & 14.5\\
	 2    & 1    & 13.9\\
	 2    & 1    & 13.6\\
	 2    & 1    & 14.4\\
	 2    & 2    & 13.5\\
	 2    & 2    & 13.1\\
	 2    & 2    & 13.3\\
	 2    & 2    & 13.1\\
	 2    & 2    & 13.4\\
\end{tabular}


    
    \subsubsection*{(a)}\label{a}

Write out the 2-way complete model for this experiment.

\[Y_{ijt} = \mu + \alpha_i + \beta_j + (\alpha \beta)_{ij} + \epsilon_{ijt}, \: \epsilon_{ijt}
\stackrel{iid}{\sim} N(0, \sigma^2)\]

\[i = 1, 2\] \[j = 1, 2\] \[t = 1, 2, 3, 4, 5\]

    \subsubsection*{(b)}\label{b}

Fit the model using R and examine the residuals. Transform the response
if needed to address any problems with normality or constant error
variance. If you transform the response, clearly show the residuals from
the un-transformed response, and your best transformation, and describe
why you chose the transformation you did.

    \paragraph{Null hypothesis:}\label{null-hypothesis}

\[H_0 : (\alpha \beta)_{ij} = 0 \: for \: all \: i, j\]

    \begin{Verbatim}[commandchars=\\\{\}]
{\color{incolor}In [{\color{incolor}90}]:} modelAB\PY{o}{=}aov\PY{p}{(}resp\PY{o}{\PYZti{}}A\PY{o}{+}B\PY{o}{+}A\PY{o}{:}B\PY{p}{,} data\PY{o}{=}df\PY{p}{)}
         anova\PY{p}{(}modelAB\PY{p}{)}
         Anova\PY{p}{(}modelAB\PY{p}{,}type\PY{o}{=}\PY{l+s}{\PYZdq{}}\PY{l+s}{III\PYZdq{}}\PY{p}{)}
\end{Verbatim}

    \begin{tabular}{|r|l|l|l|l|l|}
  & Df & Sum Sq & Mean Sq & F value & Pr(>F)\\
\hline
	A &  1           & 5.000        & 5.000        & 29.411765    & 5.632385e-05\\
	B &  1           & 0.242        & 0.242        &  1.423529    & 2.502161e-01\\
	A:B &  1           & 5.618        & 5.618        & 33.047059    & 2.991163e-05\\
	Residuals & 16           & 2.720        & 0.170        &        NA    &           NA\\
\end{tabular}

\[\]
\[\]
  
    
    \begin{tabular}{|r|l|l|l|l|}
  & Sum Sq & Df & F value & Pr(>F)\\
\hline
	(Intercept) & 727.218      &  1           & 4277.75294   & 7.312780e-21\\
	A &  10.609      &  1           &   62.40588   & 6.528729e-07\\
	B &   4.096      &  1           &   24.09412   & 1.574407e-04\\
	A:B &   5.618      &  1           &   33.04706   & 2.991163e-05\\
	Residuals &   2.720      & 16           &         NA   &           NA\\
\end{tabular}

\[\]
    
    \begin{Verbatim}[commandchars=\\\{\}]
{\color{incolor}In [{\color{incolor}91}]:} P\PYZus{}AB \PY{o}{=} \PY{l+m}{2.991163e\PYZhy{}05}
         P\PYZus{}AB \PY{o}{\PYZlt{}} \PY{l+m}{0.05}
\end{Verbatim}

    TRUE

\[\]
    
    Since the p value of \texttt{A:B} are all smaller than 0.05, hence we
are able to reject the null hypothesis at significance level of
\(\alpha = 0.05\), assume that interaction effects are not zero.

    \subsubsection*{check whether transformation is
needed}\label{check-whether-transformation-is-needed}

    \begin{Verbatim}[commandchars=\\\{\}]
{\color{incolor}In [{\color{incolor}92}]:} par\PY{p}{(}mfrow\PY{o}{=}\PY{k+kt}{c}\PY{p}{(}\PY{l+m}{2}\PY{p}{,}\PY{l+m}{2}\PY{p}{)}\PY{p}{)}
         plot\PY{p}{(}modelAB\PY{p}{)}
\end{Verbatim}

    \begin{center}
    \adjustimage{max size={0.9\linewidth}{0.9\paperheight}}{output_35_0.png}
    \end{center}
    { \hspace*{\fill} \\}
    
    From the Residual vs. Fitted plot we can see that for each vertical line
of points representing a different treatment, the spread on the points
appear to be equal. So the assumption of constant variance is not
violated.

From the QQ-plot above we could conclude that since basically all the
points fall on the dotted line, thus the residuals are normal.

Thus no transformation needed in this case.

    \subsubsection*{(c)}\label{c}

Describe the effect of Factors A and B on mean respnose. This
description should use the results of hypothesis tests and p-values as
described in class. Discuss any relevant interaction effects, main
effects and pairwise differences between treatment means. Provide a plot
that shows the means for all combinations of factor levels. Provide R
code and output that supports your results.

    Since we reject the null hypothesis and find a significant interaction
effect between A and B, so instead of looking at the levels of Factor A
and Factor B independently, we proceed directly to looking for
significant differences between all possible combinations of treatments.

    \begin{Verbatim}[commandchars=\\\{\}]
{\color{incolor}In [{\color{incolor}95}]:} lsminter\PY{o}{=}lsmeans\PY{p}{(}modelAB\PY{p}{,} \PY{o}{\PYZti{}} A\PY{o}{:}B\PY{p}{)}
         contrast\PY{p}{(}lsminter\PY{p}{,}method\PY{o}{=}\PY{l+s}{\PYZdq{}}\PY{l+s}{pairwise\PYZdq{}}\PY{p}{)}
         contrast\PYZus{}inter \PY{o}{=} cld\PY{p}{(}lsminter\PY{p}{,} alpha\PY{o}{=}\PY{l+m}{0.05}\PY{p}{)}
         contrast\PYZus{}inter
         plot\PY{p}{(}contrast\PYZus{}inter\PY{p}{)}
\end{Verbatim}
\[\]

    
    \begin{verbatim}
 contrast  estimate        SE df t.ratio p.value
 1,1 - 2,1    -2.06 0.2607681 16  -7.900  <.0001
 1,1 - 1,2    -1.28 0.2607681 16  -4.909  0.0008
 1,1 - 2,2    -1.22 0.2607681 16  -4.678  0.0013
 2,1 - 1,2     0.78 0.2607681 16   2.991  0.0389
 2,1 - 2,2     0.84 0.2607681 16   3.221  0.0247
 1,2 - 2,2     0.06 0.2607681 16   0.230  0.9955

P value adjustment: tukey method for comparing a family of 4 estimates 
    \end{verbatim}

\[\]
    
    \begin{tabular}{|r|l|l|l|l|l|l|l|l|}
  & A & B & lsmean & SE & df & lower.CL & upper.CL & .group\\
\hline
	1 & 1         & 1         & 12.06     & 0.1843909 & 16        & 11.66911  & 12.45089  &  1       \\
	4 & 2         & 2         & 13.28     & 0.1843909 & 16        & 12.88911  & 13.67089  &   2      \\
	3 & 1         & 2         & 13.34     & 0.1843909 & 16        & 12.94911  & 13.73089  &   2      \\
	2 & 2         & 1         & 14.12     & 0.1843909 & 16        & 13.72911  & 14.51089  &    3     \\
\end{tabular}


    
    
    
    \begin{center}
    \adjustimage{max size={0.7\linewidth}{0.9\paperheight}}{output_39_3.png}
    \end{center}
    { \hspace*{\fill} \\}
    
    We could interpret the results of this analysis as follows: 
    \begin{itemize}
    	\item  A1-B1 has
    	significantly lower mean than A2-B1, A1-B2, A2-B2.
    	\item  A2-B1 has
    	significantly lower mean than A1-B2, A2-B2.
    	\item  No other comparisons are
    	significantly different than zero.
    \end{itemize}
      


    % Add a bibliography block to the postdoc
    
    
    
    \end{document}
